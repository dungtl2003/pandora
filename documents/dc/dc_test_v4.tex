\documentclass[11pt,a4paper]{article}
\usepackage{times}
\usepackage[utf8]{vietnam}
\usepackage[left=3cm,right=2cm,top=2cm,bottom=2cm]{geometry} 
\usepackage{graphicx}   %thêm ảnh
\usepackage{tikz}   %vẽ
\usepackage{enumitem}   %tạo mục lục
\usepackage{listings}
\usepackage{array}  %tạo bảng có kích cỡ cụ thể
\usepackage{pgfmath} %dùng để tạo biến


\begin{document}
    \newlist{mucluc}{enumerate}{5}
    \newcommand{\mli}{\arabic{mucluci}.}
    \newcommand{\mlii}{\mli\arabic{muclucii}.}
    \newcommand{\mliii}{\mlii\arabic{mucluciii}.}
    \newcommand{\mliv}{\mliii\arabic{mucluciv}.}
    \newcommand{\mlv}{\mliv\arabic{muclucv}.}

    \setlist[mucluc,1]{
        label = \fontsize{13}{0} CHƯƠNG \mli,
        leftmargin=2.5cm,
        %rightmargin=10pt
    }
    \setlist[mucluc,2]{
        label = \mlii
    }
    \setlist[mucluc,3]{
        label = \mliii
    }
    \setlist[mucluc,4]{
        label = \mliv
    }
    \setlist[mucluc,5]{
        label = \mlv
    }

    \begin{center}
        \begin{tabular}{ c c }
            HỌC VIỆN KỸ THUẬT MẬT MÃ & \textbf{CỘNG HÒA XÃ HỘI CHỦ NGHĨA VIỆT NAM}\\
            \textbf{KHOA CÔNG NGHỆ THÔNG TIN} & \textbf{Độc lập - Tự do - Hạnh phúc}
        \end{tabular}
    \end{center}

    \begin{center}
        \fontsize{13}{35} \textbf{XÁC NHẬN ĐỀ CƯƠNG THỰC TẬP CƠ SỞ\\}
    \end{center}

    \noindent
    \hspace{-10pt}
    \begin{tabular}{ m{3.5cm} m{10cm}}
        \textbf{Tên chuyên đề: }\newline & Xây dựng trình biên dịch cho ngôn ngữ lập trình Pandora sử dụng ngôn ngữ Rust
    \end{tabular}

    \noindent
    \textbf{Lớp: } CT6D\\
    \textbf{Danh sách thành viên trong nhóm: }
    \begin{center}
        \begin{tabular}{ m{5cm} m{2.5cm} m{6cm} }
            Nguyễn Hữu Nhật Quang & 0348529273 & nhnquang01102k3@gmail.com\\
            Trần Lưu Dũng & 0966124477 & luudung0806@gmail.com
        \end{tabular}
    \end{center}

    \begin{center}
        \textbf{Nội dung đề cương chi tiết}
    \end{center}


    \begin{mucluc}
        \item TỔNG QUAN VỀ NGÔN NGỮ VÀ COMPILER
        \begin{mucluc}
            \item Giới thiệu về ngôn ngữ Pandora
            \item Khái niệm và vai trò của compiler
            \begin{mucluc}
                \item Khái niệm compiler
                \item Vai trò của compiler
            \end{mucluc}
            \item Giới thiệu về ngôn ngữ Rust
            \item Mục tiêu của đề tài
        \end{mucluc}
        \item ĐỊNH NGHĨA NGÔN NGỮ PANDORA
        \begin{mucluc}
            \item Từ tố
            \item Câu lệnh
            \item Biểu thức
            \item Đường dẫn
            \item Kiểu dữ liệu
        \end{mucluc}
        \item KIẾN TRÚC TỔNG QUAN VỀ COMPILER
        \begin{mucluc}
            \item Phân tích từ vựng
            \item Phân tích cú pháp
            \item Phân tích ngữ nghĩa
            \item Sinh mã trung gian
            \item Tối ưu mã
            \item Sinh mã
            \item Quản lý bảng ký hiệu
            \item Xử lí lỗi
        \end{mucluc}
        \item QUÁ TRÌNH XÂY DỰNG COMPILER SỬ DỤNG NGÔN NGỮ RUST
        \begin{mucluc}
            \item Thiết lập môi trường phát triển
            \item Xây dựng bảng kí hiệu
            \item Xây dựng bộ phân tích từ vựng (Lexer)
            \item Xây dựng bộ phân tích cú pháp (Parser)
            \item Xây dựng bộ phân tích ngữ nghĩa
            \item Sinh mã
            \item Xây dựng bộ xử lí lỗi
        \end{mucluc}
        \item TRIỂN KHAI THỰC NGHIỆM VÀ ĐÁNH GIÁ
        \begin{mucluc}
            \item Triển khai biên dịch chương trình in “Hello, world”
            \begin{mucluc}
                \item Quá trình biên dịch chương trình 
                \item Kết quả biên dịch chương trình
            \end{mucluc}
            \item Triển khai với những ví dụ khác
            \item Đánh giá
        \end{mucluc}
    \end{mucluc}

    % \fontsize{13}{0} 
    \noindent 
    \vspace{5pt}KẾT LUẬN VÀ HƯỚNG PHÁT TRIỂN\\\vspace{5pt}TÀI LIỆU THAM KHẢO\\PHỤ LỤC
    
    \vspace{1cm}
    \fontsize{12}{0} 
    \begin{tabular}{ m{8cm} c }
        & \textbf{Xác nhận của cán bộ hướng dẫn}\\
        & \textit{(Ký và ghi rõ họ tên)}\\
        &\\ &\\ &\\ 
        & \textbf{Nguyễn Văn Phác}
    \end{tabular}

\end{document}



