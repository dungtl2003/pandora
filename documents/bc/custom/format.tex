% Đặt mức độ hiển thị mục lục và đánh số
\setcounter{secnumdepth}{5} % Mức tối đa hiển thị khi sử dụng title
\setcounter{tocdepth}{5}    % Mức tối đa hiển thị trên mục lục

% Đặt font Times New Romans
% \fontfamily{ptm}

% Đặt khoảng cách thụt đầu đoạn mới
\setlength{\parindent}{1cm}

% Đặt khoảng cách đoạn
\setlength{\parskip}{6pt}

\titleformat{\chapter}[block]{\centering}{\textbf{\MakeUppercase\chaptername\space\thechapter.\space}}{0pt}{\textbf}{}
\titleformat{\section}[block]{}{\textbf{\thesection. }}{0pt}{\textbf}{}
\titleformat{\subsection}[block]{}{\textbf{\thesubsection. }}{0pt}{\textbf}{}
\titleformat{\subsubsection}[block]{}{\textbf{\thesubsubsection. }}{0pt}{\textbf}{}
\titleformat{\paragraph}[block]{}{\textbf{\theparagraph. }}{0pt}{\textbf}{}

\setlength{\arrayrulewidth}{0.5mm}  %set border of table
\renewcommand{\arraystretch}{1.5}   %set vertical size of table = 1.5 size default

\renewcommand{\tablename}{\itshape Bảng}
\renewcommand{\figurename}{\itshape Hình}
% \renewcommand{\thetable}{\itshape\thetable}

\setlist[itemize]{itemsep=0pt, topsep=0pt}
\addtolength{\leftmargini}{10pt}    %leftmargini add 10pt
% \addtolength{\leftmarginii}{10pt}   %leftmarginii add 10pt

\renewcommand{\listtablename}{\fontsizedefault DANH MỤC BẢNG BIỂU}
\renewcommand{\listfigurename}{\fontsizedefault DANH MỤC HÌNH VẼ}

\newcommand{\kw}[1]{\textbf{\textit{#1}}}
