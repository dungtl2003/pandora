\subsubsection{Câu lệnh biểu thức (\textit{Expression statement})}            
\label{ch2:expr_stmt}
    Câu lệnh biểu thức là câu lệnh dùng để thực thi một biểu thức nào đó. Biểu thức có thể là một biểu thức gán giá trị, một biểu thức gọi hàm hoặc một biểu thức toán tử. Câu lệnh biểu thức kết thúc bằng dấu chấm phẩy U+003B (\textbf{;}). Câu lệnh biểu thức được mô tả bởi biểu thức chính quy sau:

\regexexprstmt

\noindent Ví dụ về câu lệnh biểu thức:
\begin{lstlisting}[]
5;
4 + 3;
some_func();
\end{lstlisting}
