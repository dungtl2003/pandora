\section{Giới thiệu về ngôn ngữ Pandora}
    Tên gọi Pandora bắt nguồn từ thần thoại Hy Lạp cổ đại. Trong câu chuyện, Pandora là người phụ nữ đầu tiên được các vị thần tạo ra, người đã mở chiếc hộp chứa đựng mọi điều xấu xa tràn vào thế giới, nhưng đồng thời hy vọng vẫn còn ở lại trong đó.

    Chiếc hộp Pandora là biểu tượng của sự tò mò, những điều không thể lường trước, và cũng là khởi đầu cho những thay đổi to lớn. Nhóm chúng em chọn tên Pandora cho ngôn ngữ lập trình mới này để truyền tải ý tưởng về những thử thách, khó khăn và rủi ro tiềm ẩn khi khám phá một ngôn ngữ hoàn toàn mới. Tuy nhiên, ẩn sau đó cũng là hy vọng cho một khởi đầu mới mẻ, một hành trình đi tìm tri thức và giải pháp sáng tạo.

    Ngôn ngữ lập trình Pandora cho phép viết các biểu thức số học, bao gồm các phép toán số học thông thường như cộng ($+$), trừ ($-$), nhân ($*$), chia ($/$), chia lấy dư ($\%$), và các phép toán logic như AND, OR, NOT, XOR, \dots Ngoài ra, Pandora cũng hỗ trợ các phép dịch bit như dịch trái ($<<$) và dịch phải ($>>$), giúp lập trình viên thực hiện các thao tác trên dữ liệu nhị phân một cách hiệu quả. Pandora cũng hỗ trợ đầy đủ các toán tử so sánh như bằng ($==$), không bằng ($!=$), lớn hơn ($>$), nhỏ hơn ($<$), lớn hơn hoặc bằng ($>=$), nhỏ hơn hoặc bằng ($<=$), \dots Điều này giúp lập trình viên thực hiện các phép so sánh và kiểm tra điều kiện một cách dễ dàng và trực quan.

    Pandora là một ngôn ngữ lập trình hiện đại với cấu trúc đơn giản, linh hoạt. Ngôn ngữ này hỗ trợ các cấu trúc điều khiển như cấu trúc rẽ nhánh (when-alt), cấu trúc lặp điều khiển (while), cấu trúc lặp trình lặp qua một dãy giá trị (for), \dots giúp lập trình viên dễ dàng xây dựng các chương trình phức tạp và hiệu quả. Pandora cũng hỗ trợ khai báo và sử dụng biến, mảng, chuỗi, hàm, \dots giúp lập trình viên quản lý dữ liệu một cách linh hoạt và tiện lợi.

    Pandora có sẵn một số kiểu dữ liệu cơ bản như số nguyên (int), số thực (float), ký tự (char), chuỗi (string), mảng, \dots giúp lập trình viên dễ dàng lưu trữ và xử lý dữ liệu.

    Pandora cung cấp một hệ thống thư viện chuẩn đầy đủ, giúp lập trình viên thực hiện các thao tác nhập xuất, xử lý chuỗi, thực hiện các phép toán số học, \dots một cách nhanh chóng và tiện lợi.

    Những người lập trình viên khi mới tiếp cận một ngôn ngữ lập trình mới thường gặp khó khăn trong việc phát hiện và sửa lỗi cú pháp. Nắm bắt được vấn đề này, Pandora được thiết kế với một hệ thống thông báo lỗi cú pháp rõ ràng, chi tiết (với hơn 70 thông báo lỗi cú pháp khác nhau, kèm theo vị trí lỗi cụ thể cũng như gợi ý sửa lỗi), giúp lập trình viên dễ dàng phát hiện và sửa lỗi cú pháp một cách nhanh chóng và chính xác.

    Ngôn ngữ Pandora chỉ có vỏn vẹn 15 từ khóa, giúp giảm thiểu sự phức tạp trong quá trình học và sử dụng. Điều này giúp Pandora trở thành một ngôn ngữ lập trình lý tưởng cho người mới bắt đầu học lập trình, đồng thời cũng là công cụ hữu ích cho các lập trình viên chuyên nghiệp trong việc phát triển các ứng dụng nhỏ và trung bình. 

    Ngoài ra, để tăng thêm phần thú vị, Pandora sẽ có hai chế độ: chế độ bình thường và chế độ hỗn loạn (chaos mode). Với chế độ hỗn loạn, các từ khóa thông thường sẽ được thay thế thành các từ khóa mới mang ý nghĩa hài hước, khiến Pandora có thể trở thành một công cụ giải trí cho những lập trình viên sau những giờ làm việc căng thẳng.
