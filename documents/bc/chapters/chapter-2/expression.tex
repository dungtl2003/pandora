\section{Biểu thức (\textit{Expression})}

\regexexpr

Biểu thức là một thành phần quan trọng trong ngôn ngữ lập trình Pandora. Một biểu thức là một giá trị hoặc bất cứ thứ gì thực thi và kết thúc là một giá trị. Biểu thức có thể là một hằng số, một biến, một lời gọi hàm hay một phép toán giữa các biểu thức khác. Biểu thức có thể được sử dụng trong nhiều ngữ cảnh khác nhau như gán giá trị cho biến, truyền tham số cho hàm, điều kiện, vòng lặp, ...

Độ ưu tiên của các toán tử và biểu thức Pandora được sắp xếp như sau, từ mạnh đến yếu. 

\begin{longtable}{| l | l |}
    \caption{Bảng mức độ ưu tiên toán tử / biểu thức} \\
\hline
\textbf{\textit{Operator/Expression}} & \textbf{\textit{Associativity}} \\
\hline
Method calls & \\
\hline
Function calls & \\
\hline
\w{$-$}(unary) \w{$*$} \w{$!$} & \\
\hline
\w{$*$} \w{$/$} \w{$\%$} & left to right \\
\hline
\w{$+$} \w{$-$} & left to right \\
\hline
\w{$<<$} \w{$>>$} & left to right \\
\hline
\w{$\&$} & left to right \\
\hline
\w{$\wedge$} & left to right \\
\hline
\w{$|$} & left to right \\
\hline
\w{$==$} \w{$!=$} \w{$<$} \w{$>$} \w{$<=$} \w{$>=$} & left to right \\
\hline
\w{$\&\&$} & left to right \\
\hline
\w{$||$} & left to right \\
\hline
\w{$=$} \w{$+=$} \w{$-=$} \w{$*=$} \w{$/=$} \w{$\%=$} \w{$<<=$} \w{$>>=$} \w{$\&=$} \w{$\wedge=$} \w{$|=$} & right to left \\
\hline
\end{longtable}

\subsection{Biểu thức hằng (\textit{Literal expression})}

\regexlitexpr

Biểu thức hằng là một biểu thức mà giá trị của nó không thay đổi trong suốt quá trình thực thi chương trình. Biểu thức hằng có thể là một hằng số nguyên, hằng số thực, hằng số chuỗi hoặc hằng số boolean. Biểu thức hằng chỉ chứa đúng 1 từ tố.

\noindent\textbf{Biểu thức hằng kí tự (\textit{Character literal expression})}

1 biểu thức hằng kí tự bao gồm 1 từ tố kí tự (character literal). Kiểu của biểu thức là kiểu kí tự (char) nguyên thủy. Biểu thức hằng kí tự bắt đầu và kết thúc bằng dấu nháy đơn '\textbf{'}' (U+0027). Kí tự trong biểu thức hằng kí tự có thể là ký tự Unicode đơn hoặc ký tự thoát (escape character). 

\noindent\textbf{Biểu thức hằng chuỗi (\textit{String literal expression})}

1 biểu thức hằng chuỗi có thể là 1 từ tố chuỗi (string literal) hoặc 1 từ tố chuỗi thô (raw string literal). Kiểu của biểu thức là kiểu chuỗi (String). Nếu từ tố là chuỗi thô, nó bắt đầu bằng ký tự '\textbf{r}', tiếp theo đó có thể là 1 loạt kí tự '\textbf{\#}', rồi đến cặp dấu nháy kép '\textbf{"}' (U+0022). Chuỗi thô có thể chứa bất kỳ ký tự nào trong Unicode. Nếu từ tố là chuỗi, nó bắt đầu và kết thúc bằng dấu nháy kép '\textbf{"}' (U+0022). Chuỗi có thể chứa bất kỳ ký tự nào trong Unicode, trừ ký tự dấu nháy kép (U+0022) và ký tự thoát (U+005C).

\noindent\textbf{Biểu thức hằng số nguyên (\textit{Integer literal expression})}

1 biểu thức hằng số nguyên bao gồm 1 từ tố hằng số nguyên (integer literal). Kiểu của biểu thức là kiểu số nguyên (int) nguyên thủy. Hằng số nguyên có thể là hệ thập phân, hệ thập lục phân, hệ bát phân hoặc hệ nhị phân. Hệ thập phân bắt đầu bằng 1 chữ số thập phân, hệ thập lục phân bắt đầu bằng chuỗi '\textbf{0h}', hệ bát phân bắt đầu bằng chuỗi '\textbf{0o}' và hệ nhị phân bắt đầu bằng chuỗi '\textbf{0b}'.

\noindent\textbf{Biểu thức hằng số thực (\textit{Float literal expression})}

1 biểu thức hằng số thực bao gồm 1 từ tố hằng số thực (float literal). Kiểu của biểu thức là kiểu số thực (float) nguyên thủy. Hằng số thực có thể là số thực hoặc số thực với số mũ. Số thực bắt đầu bằng 1 số thực, số thực với số mũ bắt đầu bằng số thực theo sau là ký tự '\textbf{e}' hoặc '\textbf{E}' và tiếp theo là số nguyên. Số thực có thể có dấu chấm để phân cách phần nguyên và phần thập phân (nếu có thì phần thập phân phải trước phần mũ). 

\noindent\textbf{Biểu thức hằng số boolean (\textit{Boolean literal expression})}

1 biểu thức hằng số boolean bao gồm từ khóa \kw{true} hoặc \kw{false}. Kiểu của biểu thức là kiểu boolean (boolean) nguyên thủy.

\subsection{Biểu thức toán tử (\textit{Operator expression})}

\regexopexpr

Biểu thức toán tử là một biểu thức mà giá trị của nó phụ thuộc vào giá trị của các biểu thức con và toán tử. Biểu thức toán tử có thể là một biểu thức phủ định, biểu thức toán tử số học hoặc logic, biểu thức so sánh, biểu thức gán hoặc biểu thức boolean lười biếng. Biểu thức toán tử có thể chứa nhiều từ tố.

\noindent\textbf{Biểu thức phủ định (\textit{Negation expression})}

\regexneexpr

Biểu thức phủ định là một biểu thức mà giá trị của nó phụ thuộc vào giá trị của biểu thức con và toán tử phủ định. Biểu thức phủ định có thể là phủ định số học hoặc phủ định logic. Phủ định số học bắt đầu bằng ký tự '\textbf{-}' (U+002D), phủ định logic bắt đầu bằng ký tự '\textbf{!}' (U+0021). Ta lưu ý rằng phủ định số học chỉ áp dụng cho biểu thức số học, phủ định logic chỉ áp dụng cho biểu thức logic. Các số thực có dấu sẽ được biểu diễn bằng cách sử dụng mã bù 2.

\noindent\textbf{Biểu thức toán tử số học hoặc logic (\textit{Arithmetic or logical expression})}

\regexarithlogicexpr

Biểu thức toán tử số học hoặc logic là một biểu thức mà giá trị của nó phụ thuộc vào giá trị của các biểu thức con và toán tử số học hoặc logic. Biểu thức toán tử số học hoặc logic có thể là phép cộng, phép trừ, phép nhân, phép chia, phép chia lấy dư, phép AND, phép OR hoặc phép XOR. Các phép toán số học hoặc logic sẽ được thực hiện theo thứ tự ưu tiên. Ta cần lưu ý rằng:

\begin{itemize}
    \item{Phép chia lấy dư chỉ áp dụng cho số nguyên}
    \item{Phép AND, OR, XOR chỉ áp dụng cho số nguyên}
    \item{Phép chia giữa 2 số nguyên sẽ trả về kết quả là số nguyên (phần nguyên)}
    \item{Phép chia cho 0 sẽ sinh ra lỗi}           
\end{itemize}

\noindent\textbf{Biểu thức so sánh (\textit{Comparison expression})}

\regexcompexpr

Biểu thức so sánh là một biểu thức mà giá trị của nó luôn là từ khóa boolean \kw{true} hoặc \kw{false}. Biểu thức so sánh có thể là phép so sánh bằng, phép so sánh khác, phép so sánh nhỏ hơn, phép so sánh lớn hơn, phép so sánh nhỏ hơn hoặc bằng, phép so sánh lớn hơn hoặc bằng. Các phép so sánh sẽ trả về giá trị boolean. Cần thiết phải có cặp ngoặc tròn khi thực hiện các phép so sánh liền kề nhau. Ví dụ, \textit{a < b < c} sẽ sinh ra lỗi, cần phải sửa thành \textit{(a < b) < c}.

\noindent\textbf{Biểu thức gán (\textit{Assign expression})}

\regexassignexpr

Biểu thức gán là một biểu thức mà giá trị của nó luôn là từ khóa \kw{void}, với mục đích gán giá trị cho biến ở bên trái.

\noindent\textbf{Biểu thức gán phức tạp (\textit{Compound assignment expression})}

\regexcompoundassignexpr

Biểu thức gán phức tạp là một biểu thức mà giá trị của nó luôn là từ khóa \kw{void}, kết hợp giữa biểu thức gán và biểu thức toán tử. Biểu thức gán phức tạp có thể là phép cộng gán, phép trừ gán, phép nhân gán, phép chia gán, phép chia lấy dư gán, phép AND gán, phép OR gán hoặc phép XOR gán.

\noindent\textbf{Biểu thức boolean lười biếng (\textit{Lazy boolean expression})}

\regexlazyboolexpr

Biểu thức boolean lười biếng là một biểu thức mà giá trị của nó luôn là từ khóa boolean \kw{true} hoặc \kw{false}. Biểu thức boolean lười biếng có thể là phép AND hoặc phép OR. Gọi là lười biếng vì nó chỉ thực hiện phép toán cần thiết để xác định giá trị của biểu thức.

\subsection{Biểu thức nhóm (\textit{Grouped expression})}

\regexgroupexpr

Biểu thức nhóm là một biểu thức mà giá trị của nó phụ thuộc vào giá trị của biểu thức con. Biểu thức nhóm bắt đầu và kết thúc bằng cặp ngoặc tròn (U+0028 và U+0029). Biểu thức nhóm giúp thay đổi thứ tự thực hiện của các phép toán.

\subsection{Biểu thức lời gọi hàm (\textit{Function call expression})}

\regexfuncallexpr

Biểu thức lời gọi hàm là một biểu thức mà giá trị của nó phụ thuộc vào giá trị của hàm được gọi và các tham số truyền vào hàm. Biểu thức lời gọi hàm bắt đầu bằng tên hàm, tiếp theo là cặp ngoặc tròn (U+0028 và U+0029) chứa danh sách tham số truyền vào hàm. Danh sách tham số truyền vào hàm được phân cách bởi dấu phẩy (U+002C).

\subsection{Biểu thức lời gọi phương thức (\textit{Method call expression})}
