\section{Xây dựng bộ phân tích từ vựng}
Giai đoạn phân tích từ vựng là bước đầu tiên trong quá trình thông dịch, giúp chuyển đổi mã nguồn thành các đơn vị cơ bản gọi là từ tố. Từ tố đại diện cho các ký hiệu kết thúc như từ khóa, toán tử, tên biến, hằng, \dots, là nền tảng để bộ phân tích cú pháp xử lý tiếp theo. Việc xây dựng bộ phân tích từ vựng yêu cầu định nghĩa rõ ràng các quy tắc để nhận diện và phân loại các phần tử trong mã nguồn. Phần này sẽ trình bày quy trình triển khai bộ phân tích từ vựng để đảm bảo nhận diện chính xác các từ tố cần thiết.

Ba giai đoạn của bộ phân tích từ vựng sẽ được trình bày chi tiết như sau:

\subsection{Phân tích từ tố sơ cấp}
Trong quá trình xây dựng trình thông dịch Pandora, từ tố sơ cấp (Token) là một thành phần quan trọng giúp đại diện cho các đơn vị cơ bản nhất trong mã nguồn, chẳng hạn như từ khóa, toán tử, dấu câu, giá trị số, \dots Mỗi từ tố cần chứa thông tin về loại của nó (kind) và độ dài (len) để hỗ trợ các bước phân tích và thực thi mã nguồn sau này.

Cấu trúc của từ tố sơ cấp trong hệ thống được triển khai thông qua struct Token, bao gồm:
\begin{itemize}
  \item \textbf{kind}: TokenKind: Đại diện loại từ tố, chẳng hạn như từ khóa, số, hoặc dấu câu, \dots
  \item \textbf{len}: u32: Biểu thị độ dài của từ tố trong chuỗi mã nguồn.
\end{itemize}
Chi tiết triển khai của cấu trúc Token được trình bày tại \hyperref[ap1:simple_token]{\bf Phụ lục 1, mục 1.1.1}

Ở giai đoạn này, bộ phân tích từ vựng sẽ đọc chương trình nguồn và phân tích nó thành các từ tố sơ cấp được thể hiện qua enum \textit{lexer::token::TokenKind} (Chi tiết mã triển khai tại \hyperref[ap1:simple_token_kind]{\bf Phụ lục 1, mục 1.1.1}). Các loại từ tố sơ cấp được thể hiện như sau:

\begin{itemize}
  \item \textbf{Các ký tự đơn}: Các ký tự đơn như `:', `,', `.', `;', \dots
  \item \textbf{Các chuỗi ký tự}:
    Là một chuỗi các ký tự liên tiếp nhau tạo nên từ tố mang giá trị cụ thể và trực tiếp trong chương trình. Chuỗi ký tự sẽ có các loại được thể hiện qua enum \textit{LiteralKind} (chi tiết mã chương trình triển khai được tại \hyperref[ap1:simple_token_literal]{\bf Phụ lục 1, mục 1.1.1}), nó được thể hiện như sau:
    \begin{itemize}
      \item Chuỗi (Str). Loại từ tố này có thêm thuộc tính \textit{terminated} để xác định xem chuỗi có được đóng đúng cách hay không, ở đây chuỗi được đặt đúng cách là bên trong cặp nháy kép (").
      \item Chuỗi thô (RawStr). Loại từ tố này đại diện cho chuỗi được giữ nguyên giá trị và không bị xử lý đặc biệt bởi trình thông dịch. Điều này có nghĩa là tất cả các ký tự trong chuỗi, bao gồm cả các ký tự đặc biệt như '\textbackslash', sẽ được giữ nguyên khi chúng xuất hiện. Trong khi đó ở chuỗi thông thường, nếu muốn trình thông dịch hiểu được là '\textbackslash' thì ta cần truyền giá trị dưới dạng là '\textbackslash\textbackslash'; hoặc ví dụ khác chuỗi thông thường sẽ hiểu '\textbackslash n' là ký tự xuống dòng, nhưng trong chuỗi thô "\textbackslash n" sẽ được hiểu là 2 ký tự '\textbackslash' và 'n'; \dots\space Và nhiều ví dụ khác nữa. 
      \\Chuỗi thô sẽ được đặt trong cặp nháy kép (") bắt đầu với \textit{r}.
      \\Loại từ tố này có thêm thuộc tính \textit{n\_hashes} để đếm số kí tự '\#' nằm giữa kí tự 'r' và '"'. Nếu chuỗi bị sai, \textit{n\_hashes} sẽ mang giá trị None
      \item Số nguyên (Int). Loại từ tố này có thêm 2 thuộc tính \textit{base} và \\\textit{empty\_int}. Thuộc tính \textit{base} dùng để xác định hệ cơ số của số nguyên (ở đây Pandora hỗ trợ 4 cơ số chính là thập phân, nhị phân, bát phân và thập lục phân). Thuộc tính \textit{empty\_int} để xác định xem số nguyên có hợp lệ hay không (ví dụ 0b, 0h, 0o).
      \item Số thực (Float). Loại từ tố này có thêm 2 thuộc tính là \textit{base} và \\\textit{empty\_exponent}. Thuộc tính \textit{base} tương tự như số nguyên, nó dùng để xác định hệ cơ số của số thực nhưng Pandora chỉ hỗ trợ cơ số 10 cho số thực. Thuộc tính \textit{empty\_exponent} dùng để xác định xem số nguyên có hợp lệ hay không (phải có phần số mũ sau chữ \textit{e}, ví dụ: không hợp lệ - 314e, hợp lệ - 314e-2).
      \item Ký tự (Char). Loại từ tố này có thêm thuộc tính \textit{terminated} để xác định xem ký tự có được đóng đúng cách hay không, ở đây ký tự được đặt đúng cách là bên trong cặp nháy đơn (').
      
    \end{itemize}
  \item \textbf{Tên}: Tên biến, hàm , \dots\space và từ khóa. Từ khóa và các tên thông thường sẽ được phân biệt ở giai đoạn sau.
  \item \textbf{Tên thô}: Tương tự như tên thông thường nhưng ở đây ta có thể sử dụng từ khóa để đặt tên cho biến, hàm, \dots
  \item \textbf{Chú thích dòng}: Chú thích theo dòng. 
        Loại từ tố này có thêm thuộc tính \textit{doc\_style} dùng để xác định kiểu của chú thích là bên trong hay bên ngoài 1 khối lệnh.
\begin{lstlisting}[]
pub enum DocStyle {
  Inner,
  Outer,
}
\end{lstlisting}
  \item \textbf{Chú thích đoạn}:  Chú thích theo đoạn. 
        Loại từ tố này có thêm 2 thuộc tính là \textit{doc\_style} và \textit{terminated}. Thuộc tính \textit{doc\_style} có chức năng tương tự như chú thích dòng, để xác định kiểu của chú thích. Thuộc tính \textit{terminated} dùng để xác định xem chú thích có được đóng đúng cách không.
  \item \textbf{Khoảng trắng}. Các ký tự khoảng trắng.
  \item \textbf{Không xác định}. Các ký tự không được thể tạo thành các từ tố hợp lệ trong Pandora
  \item \textbf{Kết thúc đầu vào}.
\end{itemize}

Hàm \textit{advance\_token} (chi tiết mã triển khai tại \hyperref[ap1:simple_token_advance_token]{\bf Phụ lục 1, mục 1.1.1}) được sử dụng để đọc từ tố tiếp theo trong chuỗi ký tự đầu vào như sau:

\begin{itemize}
  \item Khoảng trắng sẽ được chuyển trực tiếp thành từ tố khoảng trắng.
  \item Các ký tự đơn như ':', ',', \dots , '|' sẽ được chuyển trực tiếp thành các loại từ tố tương ứng.
  \item Ký tự '/' đặc biệt hơn vì có thể nó sẽ là bắt đầu của 1 chú thích nên ta cần kiểm tra ký tự ngay sau nó, nếu là '/' thì đây là loại từ tố \textit{line\_comment}, nếu là '*' thì đây là loại từ tố \textit{block\_comment}, ngược lại ký tự tiếp theo này là 1 ký tự khác thì đây là loại từ tố \textit{Slash}.
  \item Nếu là số thì các loại từ tố số sẽ được xác định cụ thể là Int hay Float, cơ số nào bởi hàm \textit{number}.
  \item Xử lí loại từ tố tên thô và chuỗi thô: ký tự đang được đọc là 'r', ta sẽ xác định 2 ký tự tiếp theo ngay sau nó. Nếu ký tự tiếp theo là '\#' và sau đó là ký tự bắt đầu tên hợp lệ thì xác định loại từ tố tên thô bằng hàm \textit{raw\_identifier}; Nếu ký tự tiếp theo là '\#' hoặc '"' thì xác định loại từ tố chuỗi thô bằng hàm \textit{raw\_double\_quote\_string}; Trường hợp còn lại thì ta sẽ xác định loại từ tố tên bằng hàm \textit{identifier.}
  \item Xử lí ký tự: ký tự đang được đọc là dấu nháy đơn (') thì ta sẽ xác định loại từ tố ký tự bằng hàm \textit{single\_quote\_string}. Hàm này sẽ xử lí toàn bộ chuỗi kí tự nằm giữa cặp nháy đơn (') do đó hàm vẫn mang tên \\\textit{single\_quote\_string} mà không phải là \textit{single\_quote\_char}, nếu chuỗi này nhiều hơn 1 ký tự, chương trình sẽ báo lỗi ở giai đoạn sau.
  \item Xử lí chuỗi: ký tự đang được đọc là dấu nháy kép (") thì ta sẽ xác định loại từ tố chuỗi bằng hàm \textit{double\_quote\_string}. 
  \item Nếu ký tự đang đọc là 1 ký tự bất kì nào khác mà là ký tự bắt đầu tên hợp lệ thì ta sẽ xác định loại từ tố tên bằng hàm \textit{identifier}. 
  \item Nếu ký tự đang đọc không thỏa mãn bất kì trường hợp nào trên thì sẽ là loại từ tố không xác định.
\end{itemize}

\subsection{Phân tích từ tố thứ cấp}

Từ tố thứ cấp bao gồm các thuộc tính là loại từ tố và vị trí của từ tố:

Vị trí của từ tố thứ cấp sẽ bao gồm 2 thuộc tính là \textit{offset} cho biết vị trí bắt đầu của token trong chuỗi đầu vào, và thuộc tính \textit{length} cho biết chiều dài của token trong chuỗi đầu vào:
\begin{lstlisting}
pub struct Span {
  /// The start of the span.
  pub offset: BytePos,
  /// The total length of the span
  pub length: usize,
}
\end{lstlisting}

Ở giai đoạn này các từ tố sơ cấp sẽ được "\textit{nấu}" lên tạo thành các từ tố thứ cấp. Các loại từ tố thứ cấp được thể hiện thông qua enum \textit{ast::token::TokenKind} (chi tiết mã triển khai tại \hyperref[ap1:flex_token_kind]{\bf Phụ lục 1, mục 1.1.2}).

Cụ thể các loại từ tố thứ cấp được thể hiện như sau:
\begin{itemize}
  \item Từ tố toán tử biểu thức: `=', `<', `!', \dots, trong đó có 2 loại từ tố đặc biệt là \textit{BinOp} và \textit{BinOpEq} là loại từ tố bao gồm các ký hiệu là toán tử nhị phân (cộng, trừ, nhân, chia, chia lấy dư, and, or, xor, dịch trái, dịch phải) được thể hiện qua enum \textit{BinOpToken}:
  \item \textit{BinOp} là loại từ tố bao gồm các ký tự là phép toán (cộng, trừ, nhân, chia, chia lấy dư, and, or, xor, dịch trái, dịch phải) được thể hiện qua enum \textit{BinOpToken}:
\begin{lstlisting}
pub enum BinOpToken {
  Plus,
  Minus,
  Star,
  Slash,
  Percent,
  Caret,
  And,
  Or,
  Shl,
  Shr,
}
\end{lstlisting}
  \item Từ tố ký hiệu cấu trúc: `.', `,', `;', \dots, trong đó có 2 loại từ tố đặc biệt là \textit{OpenDelim \emph{và} CloseDelim} tương ứng thể hiện kiểu mở và đóng ngoặc. Nó được thể hiện qua enum \textit{Delimiter}:
\begin{lstlisting}
pub enum Delimiter {
  /// `( ... )`
  Parenthesis,
  /// `{ ... }`
  Brace,
  /// `[ ... ]`
  Bracket,
}        
\end{lstlisting}
  \item \textit{Literal} là loại từ tố chuỗi ký tự được thể hiện qua cấu trúc \textit{Lit} với 2 thuộc tính là \textit{kind} và \textit{symbol}. 
  \\Thuộc tính \textit{kind} có thể là một trong các loại được thể hiện qua enum \textit{LitKind}: 
\begin{lstlisting}
pub enum LitKind {
  Bool,
  Char,
  Int,
  Float,
  Str,
  RawStr(u8), // raw string delimited by `n` hash symbols

  Err,
}
\end{lstlisting}
  Thuộc tính \textit{symbol} có kiểu \textit{Symbol} là giá trị của chuỗi dưới dạng số 32bit:
\begin{lstlisting}
  pub struct Symbol(u32);
\end{lstlisting}
  \item \textit{Ident} là từ tố tên, được thể hiện bởi 2 giá trị là \textit{Symbol} lưu giá trị của tên dưới dạng số 32bit và \textit{IdentIsRaw} để xác định xem đây có phải là tên thô hay không:
\begin{lstlisting}
  pub enum IdentIsRaw {
    Yes,
    No,
  }
\end{lstlisting}
  \item \textit{DocComment} là từ tố chú thích tài liệu:
        Từ tố này thể hiện qua 3 giá trị. 
        \\Thứ nhất là \textit{CommentKind} cho biết loại chú thích là chú thích dòng hay đoạn:
\begin{lstlisting}
pub enum CommentKind {
  Line,
  Block,
}
\end{lstlisting}

        Thứ hai là \textit{Option<DocStyle>} cho biết kiểu chú thích là bên trong hay bên ngoài khối lệnh:
\begin{lstlisting}
pub enum DocStyle {
  Inner,
  Outer,
}
\end{lstlisting}

        Thứ ba là \textit{Symbol} lưu giá trị của chú thích dưới dạng số.
\end{itemize}

Hàm \textit{next\_token} (chi tiết mã nguồn triển khai được trình bày tại \hyperref[ap1:flex_token_next_token]{\bf Phụ lục 1, mục 1.1.2}) được sử dụng để đọc từ tố sơ cấp tiếp theo trong chuỗi ký tự đầu vào, sau đó "\textit{nấu}" chúng thành từ tố thứ cấp như sau:

\begin{itemize}
  \item Các khoảng trắng sẽ được bỏ qua.
  \item Từ tố thứ cấp chú thích tài liệu dòng và đoạn: Nếu từ tố đang được đọc là từ tố sơ cấp chú thích dòng hoặc đoạn thì ta thực hiện loại bỏ các chú thích thông thường, sau đó chuyển phần chú thích tài liệu còn lại thành từ tố thứ cấp chú thích tài liệu dòng hoặc đoạn tương ứng.
  \item Từ tố thứ cấp tên: Nếu từ tố đang được đọc là từ tố sơ cấp tên thì nó sẽ được chuyển thành từ tố thứ cấp tên thông qua hàm \textit{cook\_ident}; Nếu từ tố đang được đọc là từ tố sơ cấp tên thô thì nó sẽ được chuyển thành từ tố thứ cấp tên thông qua hàm \textit{cook\_raw\_ident}.
  \item Từ tố chuỗi ký tự: Nếu từ tố đang được đọc là từ tố sơ cấp chuỗi kí tự thì nó sẽ được chuyển thành từ tố thứ cấp chuỗi kí tự với kiểu tương ứng thông qua hàm \textit{cook\_literal}.
  \item Các ký tự đơn sẽ được chuyển từ từ tố sơ cấp thành từ tố thứ cấp tương ứng.
  \item Từ tố sơ cấp không xác định sẽ được bỏ qua và cảnh báo.
  \item Từ tố sơ cấp kết thúc đầu vào cũng được chuyển thành từ tố thứ cấp tương ứng.
\end{itemize}

\subsection{Phân tích cây từ tố}
Cây từ tố là cấp độ từ tố phức tạp hơn so với 2 cấp độ trước. 1 cây từ tố có thể là 1 từ tố đơn (lá), hoặc là 1 nhóm các cây từ tố con được đặt trong cặp đóng mở ngoặc (1 cây từ tố con). Điều này được thể hiện qua enum \textit{TokenTree}:

\begin{lstlisting}
pub enum TokenTree {
  /// A single token. Should never be `OpenDelim` or `CloseDelim`, because
  /// delimiters are implicitly represented by `Delimited`.
  Token(Token, Spacing),
  // A delimited sequence of token trees.
  Delimited(DelimSpan, Delimiter, TokenStream),
}
\end{lstlisting}

Cụ thể như sau:
\begin{itemize}
  \item Từ tố đơn (Token). Mỗi cây từ tố thuộc loại này bao gồm 2 giá trị: thứ nhất là giá trị thuộc \textit{Token} (từ tố thứ cấp được đoán nhận ở giai đoạn 2), thứ hai là giá trị \textit{Spacing}. \textit{Spacing} có 2 loại là \textit{Alone \emph{và} Joint}. \textit{Joint} được sử dụng khi theo sau từ tố là 1 từ tố dấu câu cho phép chúng kết hợp đựợc với nhau (ví dụ '<' và'=' có thể kết hợp thành "<=", \dots). \textit{Alone} được sử dụng cho các loại từ tố còn lại. 
\begin{lstlisting}
pub enum Spacing {
  Alone,
  Joint,
}
\end{lstlisting}
  \item Nhóm các cây từ tố con được đặt trong cặp ngoặc (Delimited). Mỗi cây từ tố thuộc loại này bao gồm 3 giá trị:
  \begin{itemize}
    \item DelimSpan: vị trí mở ngoặc và đóng ngoặc
\begin{lstlisting}
  pub struct DelimSpan {
    pub open: Span,
    pub close: Span,
  }
\end{lstlisting}
    \item Delimiter: loại của ngoặc (ngoặc ngọn, ngoặc tròn, ngoặc vuông)
\begin{lstlisting}
  pub enum Delimiter {
    /// `( ... )`
    Parenthesis,
    /// `{ ... }`
    Brace,
    /// `[ ... ]`
    Bracket,
  }
\end{lstlisting}
    \item TokenStream: 1 danh sách các cây từ tố con
  \end{itemize}
\end{itemize}

Hàm \textit{lex\_token\_trees} (chi tiết mã nguồn triển khai được trình bày tại \hyperref[ap1:token_tree_lex_token_trees]{\bf Phụ lục 1, mục 1.1.3}) sẽ đọc các từ tố thứ cấp nhận được từ giai đoạn 2 và chuyển chúng thành cây từ tố như sau:

\begin{itemize}
  \item Nếu từ tố đang được đọc là dấu mở ngoặc, ta sẽ bắt đầu đoán nhận 1 cây từ tố loại \textit{Delimited} thông qua hàm \textit{lex\_token\_tree\_open\_delim}, đồng thời nếu trong quá trình đoán nhận này nếu có xuất hiện 1 cây từ tố con không được đóng đúng cách:
  \item Nếu từ tố đang được đọc là dấu đóng ngoặc, sẽ trả về cây từ tố đã đọc được, đồng thời sẽ đưa ra cảnh báo nếu cây từ tố này không được đóng đúng cách.
  \item Nếu từ tố đang được đọc là Từ tố kết thúc đầu vào, sẽ trả về cây từ tố đã đọc được, đồng thời sẽ đưa ra cảnh báo nếu cây từ tố này không được đóng đúng cách.
  \item Nếu từ tố đang được đọc là từ tố thứ cấp bình thường, sẽ thêm cây từ tố loại \textit{Token} vào danh sách cây từ tố hiện tại.
\end{itemize}