\section{Xây dựng bộ phân tích cú pháp}
\label{ch3:syntax-analysis}
Giai đoạn phân tích cú pháp là bước tiếp theo sau khi mã nguồn đã được chuyển đổi thành các token bởi bộ phân tích từ vựng. Mục tiêu của giai đoạn này là xây dựng cây cú pháp trừu tượng (AST) để biểu diễn cấu trúc logic của mã nguồn, tạo nền tảng cho các giai đoạn phân tích ngữ nghĩa và thực thi sau này. 

Việc xây dựng bộ phân tích cú pháp đòi hỏi một quy trình rõ ràng nhằm đảm bảo trình thông dịch có thể nhận diện đúng cú pháp và xử lý chính xác các câu lệnh phức tạp.

Trong ngôn ngữ Pandora, các loại câu lệnh như khai báo biến, biểu thức, câu lệnh điều khiển, và các cấu trúc phức tạp khác đóng vai trò quan trọng trong việc xây dựng logic của chương trình. Việc triển khai bộ phân tích cú pháp cho các câu lệnh này là một phần thiết yếu trong việc xây dựng trình thông dịch, giúp đảm bảo rằng tất cả các cấu trúc cú pháp được phân tích và xử lý đúng cách. Phần này sẽ giới thiệu cách thức bộ phân tích cú pháp hoạt động khi phân tích các loại câu lệnh khác nhau, cùng với việc giải thích chi tiết các mã nguồn triển khai được cung cấp tại phần phụ lục.

\subsection{Câu lệnh}
Quá trình phân tích cú pháp bắt đầu bằng việc nhận diện loại câu lệnh hiện tại. Điều này được thực hiện thông qua từ tố tiếp theo trong luồng đầu vào và được xử lý bởi hàm \textit{parse\_stmt}. Hàm này có nhiệm vụ phân loại câu lệnh dựa trên từ tố đầu tiên, từ đó chuyển hướng quá trình phân tích đến hàm phù hợp cho từng loại câu lệnh (ví dụ: khai báo biến, câu lệnh điều kiện, vòng lặp, v.v.).

Chi tiết mã nguồn của hàm \textit{parse\_stmt} sẽ được trình bày tại \hyperref[ap1:stmt]{\bf Phụ lục 1}. % Mục 1.xx todo!


\subsubsection{Câu lệnh khai báo}

\noindent\textbf{Câu lệnh khai báo biến:}

Cú pháp của câu lệnh khai báo biến được thể hiện thông qua biểu thức chính quy tại \ref{ch2:decl_var_stmt}.

Câu lệnh khai báo biến được bắt đầu khi từ tố hiện tại là từ khóa \kw{set}. Việc phân tích câu lệnh này được thực hiện thông qua hàm \textit{parse\_stmt\_var\_decl}, chi tiết mã nguồn triển khai được trình bày trong \hyperref[ap1:stmt_decl_var]{\bf Phụ lục 1}. %, Mục 1.xx todo!

Cụ thể hàm \textit{parse\_stmt\_var\_decl} chịu trách nhiệm triển khai các bước phân tích cú pháp câu lệnh khai báo biến như sau:

\begin{itemize}
    \item \textbf{Xác định chế độ gán giá trị của biến:}
    Sau khi nhận diện từ khóa \kw{set}, trình thông dịch kiểm tra từ khóa bổ sung để xác định liệu biến được khai báo có thể thay đổi giá trị hay không. Nếu từ khóa \kw{mut} xuất hiện ngay sau \kw{set}, biến được đánh dấu là mutable (có thể thay đổi giá trị). Ngược lại, biến sẽ là immutable (không thể thay đổi giá trị).
    \item \textbf{Xác định tên biến:}
    Sau khi đã xác định được chế độ gán giá trị của biến, trình thông dịch sẽ đọc từ tố tiếp theo từ luồng đầu vào và kiểm tra để đảm bảo nó là một tên hợp lệ. Nếu tên không hợp lệ thì báo lỗi. 
    \item \textbf{Xác định kiểu dữ liệu của biến:}
    Sau khi xác định tên biến, trình thông dịch kiểm tra từ tố tiếp theo có phải là kí tự ':' và xác định kiểu dữ liệu được chỉ định ngay sau nó bằng cách gọi hàm \textit{parse\_ty}. Nếu kiểu dữ liệu không hợp lệ thì báo lỗi. 
    \item \textbf{Kiểm tra, xử lý giá trị khởi tạo (nếu có) và phân loại:}
    Sau khi xác định kiểu dữ liệu, trình thông dịch sẽ kiểm tra từ tố tiếp theo. Nếu nó không phải là 1 dấu '=' thì trình thông dịch xác định loại của câu lệnh khai báo là \textit{Declaration} (tức chỉ khai báo mà không gán giá trị). Ngược lại nếu nó là dấu '=' thì trình thông dịch sẽ xác định biểu thức khởi tạo ngay sau nó bằng cách gọi hàm \textit{parse\_expr} và xác định loại của câu lệnh khai báo là \textit{Initialization} (tức là câu lệnh khai báo có gán giá trị), nhưng nếu biểu thức này không hợp lệ thì báo lỗi. 
    \item \textbf{Đảm bảo câu lệnh kết thúc hợp lệ:}
    Câu lệnh khai báo kết thúc bởi một kí tự ';', do đó sau khi đã xác định xong các thành phần cơ bản của câu lệnh khai báo, trình thông dịch sẽ phải kiểm tra để đảm bảo tồn tại kí tự ';' để kết thúc câu lệnh nếu như không tồn tại kí tự ';' thì báo lỗi.
\end{itemize}


\noindent\textbf{Câu lệnh khai báo hàm:}

Cú pháp của câu lệnh khai báo biến được thể hiện thông qua biểu thức chính quy tại phần \textbf{\ref{ch2:decl_func_stmt}}.

Câu lệnh khai báo hàm được bắt đầu khi từ tố hiện tại là từ khóa \kw{fun}. Việc phân tích câu lệnh này được thực hiện thông qua hàm \textit{parse\_stmt\_func\_decl}, chi tiết mã nguồn triển khai được trình bày trong \hyperref[ap1:stmt_decl_fun]{\bf Phụ lục 1}. % Mục 1.xx todo!

Cụ thể hàm \textit{parse\_stmt\_func\_decl} chịu trách nhiệm triển khai các bước phân tích cú pháp câu lệnh khai báo hàm như sau:

\begin{itemize}
    \item \textbf{Xác định phần ký hiệu của hàm}: Sau khi nhận diện từ khóa \kw{fun}, trình thông dịch bắt đầu phân tích phần ký hiệu của hàm bằng cách gọi hàm \textit{parse\_stmt\_func\_sig} (chi tiết mã nguồn triển khai tại \hyperref[ap1:stmt_decl_fun_sig]{\bf Phụ lục 1}. % Mục 1.xx todo! 
    Phần ký hiệu này bao gồm:
    \begin{itemize}
        \item Tên hàm. Một tên hợp lệ với quy tắc đặt tên của Pandora
        \item Tham số truyền vào. Sau khi xác định tên hàm, danh sách các tham số truyền vào được xác định thông qua chuỗi từ tố được đặt bên trong cặp ngoặc `(' và `)'. Mỗi tham số sẽ được phân tích lần lượt như sau: đầu tiên nếu trình thông dịch đọc được từ khóa \kw{mut} thì xác định tham số đó có thể thay đổi giá trị và ngược lại, tiếp đến trình thông dịch xác định tên tham số, cuối cùng là xác định kiểu dữ liệu của tham số được ngăn cách với tên tham số bằng ký tự `:'. Mỗi tham số phân cách nhau bởi ký tư `,'.
        \item Kiểu dữ liệu trả về. Sau khi xác định tham số truyền vào, nếu từ tố đọc được là từ tố "->" thì trình thông dịch sẽ xác định kiểu dữ liệu trả về của hàm thông qua từ tố tiếp theo.
    \end{itemize}
    \item \textbf{Xác định phần thân hàm}: Sau khi xác định được phần ký hiệu, trình thông dịch sẽ phân tích câu lệnh phần thân hàm bằng cách gọi hàm \textit{parse\_stmt}.
\end{itemize}

\subsubsection{Câu lệnh khối}

Cú pháp của câu lệnh khối được thể hiện thông qua biểu thức chính quy tại \hyperref[ch2:block_stmt]{\bf Chương 2}.

Câu lệnh khối được bắt đầu khi từ tố đọc được hiện tại là kí tự `\{'. Việc phân tích câu lệnh này được thực hiện thông qua hàm \textit{parse\_stmt\_block}, chi tiết mã nguồn triển khai được trình bày trong \hyperref[ap1:stmt_block]{\bf Phụ lục 1}. % , Mục 1.xx todo!

Thông qua hàm \textit{parse\_stmt\_block}, từng câu lệnh sau kí tự `\{' sẽ được phân tích cho đến khi gặp từ tố đóng ngoặc nhọn `\}'.

\subsubsection{Câu lệnh điều khiển}

\noindent\textbf{Câu lệnh rẽ nhánh:} 

Cú pháp của câu lệnh rẽ nhánh được thể hiện thông qua biểu thức chính quy tại \hyperref[ch2:if_stmt]{\bf Chương 2}.

Câu lệnh rẽ nhánh được bắt đầu khi từ tố đọc được hiện tại là từ khóa \kw{when}. Việc phân tích câu lệnh này được thực hiện thông qua hàm \textit{parse\_stmt\_if}, chi tiết mã nguồn triển khai được trình bày trong \hyperref[ap1:stmt_when]{\bf Phụ lục 1}. %Mục 1.xx todo!

Cụ thể hàm \textit{parse\_stmt\_if} chịu trách nhiệm triển khai các bước phân tích cú pháp câu lệnh rẽ nhánh như sau:
\begin{itemize}
    \item \textbf{Phân tích biểu thức điều kiện}: 
    Sau khi nhận diện từ khóa \kw{when}, trình thông dịch tiến hành phân tích cú pháp biểu thức theo sau bằng cách gọi hàm \textit{parse\_expr}. Nếu biểu thức này không hợp lệ thì báo lỗi.
    \item \textbf{Phân tích câu lệnh của nhánh \kw{when}}: 
    Sau khi phân tích biểu thức điều kiện, trình thông dịch gọi hàm \textit{parse\_stmt} để phân tích cú pháp câu lệnh nhánh \kw{when}.
    \item \textbf{Xử lí tùy chọn nhánh \kw{alt} nếu có}:
    Sau khi phân tích câu lệnh của nhánh \kw{when}, nếu từ tố tiếp theo là từ khóa \kw{alt} thì trình thông dịch sẽ tiếp tục phân tích cú pháp câu lệnh tiếp theo bằng cách gọi hàm \textit{parse\_stmt}.
\end{itemize}


\noindent\textbf{Câu lệnh lặp biểu thức điều kiện:}

Cú pháp của câu lệnh lặp biểu thức điều kiện được thể hiện thông qua biểu thức chính quy tại \hyperref[ch2:while_stmt]{\bf Chương 2}.

Câu lệnh lặp biểu thức điều kiện được bắt đầu khi từ tố đọc được hiện tại là từ khóa \kw{during}. Việc phân tích câu lệnh này được thực hiện thông qua hàm \textit{parse\_stmt\_while}, chi tiết mã nguồn triển khai được trình bày trong \hyperref[ap1:stmt_during]{\bf Phụ lục 1}. % , Mục 1.xx todo!

Cụ thể hàm \textit{parse\_stmt\_while} chịu trách nhiệm triển khai các bước phân tích cú pháp câu lệnh lặp biểu thức điều kiện như sau:
\begin{itemize}
    \item \textbf{Phân tích biểu thức điều kiện}:
    Sau khi nhận diện từ khóa \kw{during}, trình thông dịch tiến hành phân tích biểu thức theo sau bằng cách gọi hàm \textit{parse\_expr}. Nếu biểu thức này không hợp lệ thì báo lỗi.
    \item \textbf{Phân tích câu lệnh sau biểu thức điều kiện}:
    Sau khi phân tích biểu thức điều kiện, trình thông dịch gọi hàm \textit{parse\_stmt} để phân tích cú pháp câu lệnh tiếp theo.
\end{itemize}

\noindent\textbf{Câu lệnh lặp trình lặp:} 

Cú pháp của câu lệnh lặp trình lặp được thể hiện thông qua biểu thức chính quy tại \hyperref[ch2:for_stmt]{\bf Chương 2}.

Câu lệnh lặp trình lặp được bắt đầu khi từ tố đọc được hiện tại là từ khóa \kw{for}. Việc phân tích câu lệnh này được thực hiện thông qua hàm \textit{parse\_stmt\_for}, chi tiết mã nguồn triển khai được trình bày trong \hyperref[ap1:stmt_for]{\bf Phụ lục 1}. % , Mục 1.xx todo!

Cụ thể hàm \textit{parse\_stmt\_for} chịu trách nhiệm triển khai các bước phân tích cú pháp câu lệnh lặp trình lặp như sau:

\begin{itemize}
    \item \textbf{Phân tích tên biến sử dụng cho trình lặp}:
    Sau khi nhận diện từ khóa \kw{for}, trình thông dịch tiến hành đọc từ tố tiếp theo từ luồng đầu vào và kiểm tra để đảm bảo nó là một tên hợp lệ. Nếu tên không hợp lệ thì báo lỗi.
    \item \textbf{Xác định từ khóa \kw{in}}:
    Sau khi xác định tên biến lặp, trình thông dịch tiến hành kiểm tra từ tố tiếp theo có phải là từ khóa \kw{in} hay không. Nếu không phải từ khóa \kw{in} thì báo lỗi.
    \item \textbf{Phân tích biểu thức trình lặp}:
    Sau khi xác định được từ khóa \kw{in}, trình thông dịch tiến hành gọi hàm \textit{parse\_expr} để phân tích biểu thức theo sau. Nếu biểu thức này không hợp lệ thì báo lỗi.
    \item \textbf{Phân tích câu lệnh lặp}:
    Sau khi phân tích biểu thức điều kiện, trình thông dịch gọi hàm \textit{parse\_stmt} để phân tích cú pháp câu lệnh của trình lặp.
\end{itemize}

\noindent\textbf{Câu lệnh trả về:}

Cú pháp của câu lệnh trả về được thể hiện thông qua biểu thức chính quy tại \hyperref[ch2:return_stmt]{\bf Chương 2}.

Câu lệnh trả về được bắt đầu khi từ tố đọc được hiện tại là từ khóa \kw{yeet}. Việc phân tích câu lệnh này được thực hiện thông qua hàm \textit{parse\_stmt\_return}, chi tiết mã nguồn triển khai được trình bày trong \hyperref[ap1:stmt_yeet]{\bf Phụ lục 1}. % , Mục 1.xx todo!

Cụ thể hàm \textit{parse\_stmt\_return} chịu trách nhiệm triển khai các bước phân tích cú pháp câu lệnh trả về như sau: 
Sau khi nhận diện từ khóa \kw{yeet}, trình thông dịch sẽ phân tích xem câu lệnh trả về có bao gồm giá trị hay không dựa trên từ tố theo sau từ khóa \kw{yeet}. Nếu từ tố này có thể bắt đầu một biểu thức thì trình thông dịch sẽ bắt đầu phân tích biểu thức và xác định câu lệnh trả về có giá trị. Nếu không, từ tố này bắt buộc phải là ký tự `;' để kết thúc và xác định câu lệnh trả về không có giá trị, nếu không sẽ báo lỗi.



\noindent\textbf{Câu lệnh thoát khỏi vòng lặp:}

Cú pháp của câu lệnh thoát khỏi vòng lặp được thể hiện thông qua biểu thức chính quy tại \hyperref[ch2:break_stmt]{\bf Chương 2}.

Câu lệnh thoát khỏi vòng lặp được bắt đầu khi từ tố đọc được hiện tại là từ khóa \kw{br}. Việc phân tích câu lệnh này được thực hiện thông qua hàm \textit{parse\_stmt\_break}, chi tiết mã nguồn triển khai được trình bày trong \hyperref[ap1:stmt_br]{\bf Phụ lục 1}. % , Mục 1.xx todo!

Cụ thể hàm \textit{parse\_stmt\_break} chịu trách nhiệm triển khai các bước phân tích cú pháp câu lệnh thoát khỏi vòng lặp như sau: Sau khi nhận diện từ khóa \kw{br}, trình thông dịch sẽ nhận diện từ tố tiếp theo phải là ký tự `;' nếu không sẽ báo lỗi.


\noindent\textbf{Câu lệnh tiếp tục vòng lặp:}

Cú pháp của câu lệnh tiếp tục vòng lặp được thể hiện thông qua biểu thức chính quy tại \hyperref[ch2:continue_stmt]{\bf Chương 2}.

Câu lệnh tiếp tục vòng lặp được bắt đầu khi từ tố đọc được hiện tại là từ khóa \kw{skip}. Việc phân tích câu lệnh này được thực hiện thông qua hàm \textit{parse\_stmt\_continue}, chi tiết mã nguồn triển khai được trình bày trong \hyperref[ap1:stmt_skip]{\bf Phụ lục 1}. % , Mục 1.xx todo!

Cụ thể hàm \textit{parse\_stmt\_continue} chịu trách nhiệm triển khai các bước phân tích cú pháp câu lệnh tiếp tục vòng lặp như sau:
Sau khi nhận diện từ khóa \kw{skip}, trình thông dịch sẽ nhận diện từ tố tiếp theo phải là ký tự `;' nếu không sẽ báo lỗi.

\subsubsection{Câu lệnh thêm thư viện}

Cú pháp của câu lệnh thêm thư viện được thể hiện thông qua biểu thức chính quy tại \hyperref[ch2:import_stmt]{\bf Chương 2}.

Câu lệnh thêm thư viện được bắt đầu khi từ tố đọc được hiện tại là từ khóa \kw{add}. Việc phân tích câu lệnh này được thực hiện thông qua hàm \textit{parse\_stmt\_import}, chi tiết mã nguồn triển khai được trình bày trong \hyperref[ap1:stmt_add]{\bf Phụ lục 1}. % , Mục 1.xx todo!

Cụ thể hàm \textit{parse\_stmt\_import} chịu trách nhiệm triển khai các bước phân tích cú pháp câu lệnh thêm thư viện như sau: 
sau khi nhận diện từ khóa \kw{add}, trình thông dịch sẽ phân tích đường dẫn đến thư viện bằng cách kiểm tra xem từ tố tiếp theo có phải là một tên hợp lệ không.

\subsubsection{Câu lệnh rỗng}

Cú pháp của câu lệnh rỗng được thể hiện thông qua biểu thức chính quy tại \hyperref[ch2:empty_stmt]{\bf Chương 2}.

Câu lệnh rỗng được bắt đầu khi từ tố đọc được hiện tại là ký tự `;'. Việc phân tích câu lệnh này được thực hiện thông qua hàm \textit{parse\_stmt\_empty}, chi tiết mã nguồn triển khai được trình bày trong \hyperref[ap1:stmt_semicolon]{\bf Phụ lục 1}.

Thông qua hàm \textit{parse\_stmt\_empty}, ký tự `;' được đọc và xác định câu lệnh rỗng, nếu không tìm thấy ký tự `;' sẽ báo lỗi.

\subsubsection{Câu lệnh biểu thức:} 
Cú pháp của câu lệnh biểu thức được thể hiện thông qua biểu thức chính quy tại \hyperref[ch2:expr_stmt]{\bf Chương 2}.

Câu lệnh biểu thức được bắt đầu khi từ tố đọc được hiện tại có thể bắt đầu một biểu thức. Việc phân tích câu lệnh này được thực hiện thông qua hàm \textit{parse\_stmt\_expr}, chi tiết mã nguồn triển khai được trình bày trong \hyperref[ap1:stmt_expr]{\bf Phụ lục 1}. % , Mục 1.xx todo!

Thông qua hàm \textit{parse\_stmt\_expr}, biểu thức sẽ được trình thông dịch phân tích bằng cách gọi đến hàm \textit{parse\_expr} (cụ thể sẽ được trình bày bên dưới). Sau khi phân tích được biểu thức, trình thông dịch sẽ kiểm tra từ tố tiếp theo có phải là ký tự `;', nếu không sẽ báo lỗi.

\subsection{Biểu thức}
Cú pháp của biểu thức được thể hiện thông qua biểu thức chính quy tại \hyperref[ch2:expr]{\bf Chương 2}.

Quá trình phân tích biểu thức sẽ được thực hiện thông qua hàm \textit{parse\_expr}, chi tiết mã nguồn triển khai được trình bày trong \hyperref[ap1:expr]{\bf Phụ lục 1}. % , Mục 1.xx todo!

Đầu tiên trình thông dịch sẽ phân tích phần tiền tố của toán tử biểu thức đầu tiên thông qua hàm \textit{parse\_expr\_prefix}, chi tiết mã nguồn triển khai tại \hyperref[ap1:expr_prefix]{\bf Phụ lục 1}:
\begin{itemize}
    \item \textbf{Phân tích phần tiền tố}. Hàm \textit{parse\_expr\_prefix} kiểm tra và trả về kiểu của tất cả các từ tố là \textbf{Not}('!') hoặc là \textbf{Minus}('-').
    \item \textbf{Phân tích phần biểu thức theo sau tiền tố}. Sau khi đã bỏ qua phần tiền tố, trình thông dịch sẽ xử lý biểu thức tiếp theo thông qua hàm \textit{parse\_expr\_dot\_or\_call} (chi tiết mã nguồn triển khai được trình bày trong \hyperref[ap1:expr_dot_or_call]{\bf Phụ lục 1}). Hàm này sẽ phân tích biểu thức có mức ưu tiên cao nhất theo sau tiền tố thông qua hàm \textit{parse\_expr\_bottom}, sau đó trình thông dịch sẽ phân tích phần biểu thức đặc biệt (nếu có) thông qua hàm \textit{parse\_expr\_dot\_or\_call\_with}. Biểu thức đặc biệt cần xử lý riêng này có thể là biểu thức truy cập thư viện (Library access) hoặc biểu thức gọi hàm (Function calls) đã được đề cập đến trong bảng \ref{table2:precedence}.
    \item \textbf{Phân tích phần toán tử có mức độ ưu tiên cao nhất theo sau toán tử một ngôi}. Hàm \textit{parse\_expr\_bottom} (chi tiết mã nguồn triển khai được trình bày trong \hyperref[ap1:expr_bottom]{\bf Phụ lục 1}) phân tích biểu thức ngay sau phần tiền tố. Nếu từ tố hiện tại là tên biến hoặc là giá trị trực tiếp thì sẽ được phân tích thông qua các hàm tương ứng, nếu đây là ký tự `(' thì trình thông dịch sẽ bắt đầu phân tích 1 biểu thức mới, nếu đây là ký tự `[' thì trình thông dịch sẽ bắt đầu phân tích biểu thức mảng. Nếu không thỏa mãn trường hợp nào sẽ báo lỗi.
    \item Hàm \textit{parse\_expr\_dot\_or\_call\_with} có chi tiết mã nguồn triển khai được trình bày trong \hyperref[ap1:expr_dot_or_call_with]{\bf Phụ lục 1} sẽ phân tích từng biểu thức dựa trên từ tố hiện tại là `.' hay `(' hay `['.
    \begin{itemize}
        \item \textbf{Nếu từ tố này là ký tự `.'}, trình thông dịch sẽ phân tích biểu thức truy cập thư viện thông qua hàm \textit{parse\_expr\_dot} (chi tiết mã nguồn triển khai tại \hyperref[ap1:expr_dot]{Phụ lục 1}). Từ tố tên tiếp theo sẽ được trình thông dịch phân tích và trả về biểu thức truy cập thư viện, nếu không tìm thấy từ tố tên sẽ báo lỗi.
        \item \textbf{Nếu từ tố này là ký tự `('}, trình thông dịch sẽ phân tích biểu thức gọi hàm thông qua hàm \textit{parse\_expr\_call\_with} (chi tiết mã nguồn triển khai tại \hyperref[ap1:expr_call_with]{Phụ lục 1}). Trình thông dịch sẽ phân tích các biểu thức bên trong cặp ngoặc `(', `)' và kiểm tra phần biểu thức trước nó để xác định xem đây là biểu thức gọi hàm bình thường hay là gọi hàm của thư viện. Nếu có biểu thức không hợp lệ sẽ báo lỗi.
        \item \textbf{Nếu từ tố này là ký tự `['}, trình thông dịch sẽ phân tích biểu thức ký tự mảng thông qua hàm \textit{parse\_expr\_array\_index} (chi tiết mã nguồn triển khai tại \hyperref[ap1:expr_array_index]{Phụ lục 1}). Trình thông dịch sẽ phân tích biểu thức nằm bên trong cặp ngoặc `[', `]', nếu biểu thức khong hợp lệ sẽ báo lỗi.
    \end{itemize}
    Điều này sẽ được lặp lại cho đến khi không tìm thấy một trong ba từ tố này.
\end{itemize}
Sau đó, trình thông dịch sẽ phân tích phần còn lại của biểu thức thông qua hàm \textit{parse\_expr\_rest}. Với các mức độ ưu tiên của toán tử được đề cập đến trong bảng \ref{table2:precedence}, cụ thể hàm này thực hiện việc phân tích phần còn lại của biểu thức như sau:
\begin{itemize}
    \item Lấy ra toán tử hiện tại và xác định thứ tự ưu tiên của nó so với toán tử trước đó, nếu không có toán tử thích hợp hoặc toán tử có mức độ ưu tiên thấp hơn thì dừng lại để trả về biểu thức trước đó.
    \item Nếu toán tử hiện tại có mức độ ưu tiên cao hơn, trình thông dịch sẽ tiếp tục phân tích toán hạng tiếp theo từ phần tiền tố thông qua hàm \textit{parse\_expr\_prefix} đã được trình bày phía trên.
    \item Sau khi đã phân tích toán hạng tiếp theo, trình thông dịch sẽ tiếp tục tính toán mức độ ưu tiên cho nó và phân tích phần còn lại bằng cách gọi đệ quy hàm \textit{parse\_expr\_rest}.
    \item Sau khi đến được mức có độ ưu tiên cao nhất sẽ xác định loại của biểu thức hiện tại.
\end{itemize}
Quá trình này sẽ lặp lại cho đến khi đọc hết biểu thức hoặc khi phân tích 1 biểu thức nào đó không hợp lệ.
