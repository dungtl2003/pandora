\begin{center}
    \bf{LỜI NÓI ĐẦU}
\end{center}

\newlist{mucluc}{enumerate}{1}
\newcommand{\mli}{\arabic{mucluci}.}
% \newcommand{\mlii}{\mli\arabic{muclucii}.}


\setlist[mucluc,1]{
    label = \bf{CHƯƠNG \mli},
    leftmargin=3cm,
    before={\setlength{\topsep}{0cm}},
    % after=\vspace*{-1cm}
    %rightmargin=10pt
}
% \setlist[mucluc,2]{
%     label = \mlii
% }

Trong bối cảnh phát triển mạnh mẽ của ngành công nghệ thông tin, ngôn ngữ lập trình và 
công cụ biên dịch (compiler) đóng vai trò quan trọng trong việc xây dựng các phần mềm 
hiện đại. Từ các công cụ phân tích dữ liệu đến các hệ thống điều khiển công nghiệp, 
các ngôn ngữ lập trình mới không ngừng được ra đời nhằm đáp ứng nhu cầu đa dạng và 
chuyên biệt của ngành.

Nhóm chúng em thực hiện đề tài "Xây dựng trình biên dịch cho ngôn ngữ lập trình Pandora 
sử dụng ngôn ngữ Rust" nhằm nghiên cứu, thiết kế ngôn ngữ lập trình Pandora và triển khai 
một trình biên dịch dành riêng cho ngôn ngữ này. Đề tài này được lựa chọn với mục tiêu 
khám phá các khía cạnh cốt lõi của quá trình biên dịch và mở ra hướng đi mới cho việc 
ứng dụng Rust vào các dự án phức tạp.

Qua quá trình thực hiện, nhóm đã tìm hiểu về kiến trúc của compiler, từ các giai đoạn 
phân tích từ vựng, phân tích cú pháp, phân tích ngữ nghĩa, đến sinh mã và tối ưu mã. 
Đồng thời, chúng em đã triển khai các kỹ thuật và thư viện mới của Rust nhằm đạt được 
hiệu quả cao trong việc biên dịch và xử lý lỗi.

Nội dung báo cáo được chia làm bốn chương cụ thể như sau:

\begin{mucluc}
    
    \item {\bf{Tổng quan về ngôn ngữ và compiler}} - 
        Giới thiệu về nguồn gốc và ý nghĩa của ngôn ngữ lập trình Pandora và các khái niệm cơ bản liên quan đến trình biên dịch (compiler). Em sẽ trình bày vai trò và tầm quan trọng của compiler trong quá trình phát triển phần mềm. Đồng thời, chương này cũng cung cấp cái nhìn tổng quát về ngôn ngữ lập trình Rust, ngôn ngữ được sử dụng để xây dựng trình biên dịch cho ngôn ngữ lập trình Pandora, và lý do lựa chọn Rust cho dự án. Và phần cuối của chương sẽ là mục tiêu của đề tài do bọn em đặt ra.
    \item {\bf{Định nghĩa ngôn ngữ Pandora}} - 
        Mô tả chi tiết ngôn ngữ lập trình Pandora, giúp làm rõ hơn về các khía cạnh cấu trúc và cú pháp của ngôn ngữ. Chương này phân tích cách thức tổ chức mã nguồn, quy tắc cú pháp các câu lệnh khai báo, câu lệnh điều khiển, câu lệnh biểu thức, \dots và từ khóa quan trọng, đồng thời trình bày các kiểu dữ liệu cơ bản và phức tạp, cách sử dụng biểu thức và toán tử. Những kiến thức này tạo nền tảng cho các chương tiếp theo trong việc xây dựng trình biên dịch.
    \item {\bf{Kiến trúc tổng quan về compiler}} - 
        Phân tích các thành phần chính cấu thành nên một compiler, 
        bao gồm các bước quan trọng trong quá trình biên dịch: phân tích 
        từ vựng (lexical analysis), phân tích cú pháp (syntax analysis), 
        phân tích ngữ nghĩa (semantic analysis), sinh mã trung gian, 
        tối ưu mã và sinh mã đích. Ngoài ra, chương này còn mô tả vai 
        trò của bảng ký hiệu và cách thức xử lý lỗi trong quá trình biên 
        dịch.
    \item {\bf{Quá trình xây dựng compiler sử dụng ngôn ngữ Rust}} - 
        Trình bày chi tiết từng bước trong quá trình xây dựng compiler 
        Pandora bằng Rust. Em sẽ giới thiệu cách thiết lập môi trường 
        phát triển, sau đó đi vào các phần chính của compiler như: 
        xây dựng bảng ký hiệu, xây dựng bộ phân tích từ vựng (Lexer), 
        bộ phân tích cú pháp (Parser), bộ phân tích ngữ nghĩa, và cuối 
        cùng là sinh mã và xử lý lỗi. Các bước này sẽ được mô tả kỹ càng 
        để làm rõ quy trình triển khai và các kỹ thuật đặc thù trong Rust.
    \item {\bf{Triển khai, thực nghiệm và đánh giá}} - 
        trình bày các thử nghiệm thực tế nhằm đánh giá khả năng và hiệu quả 
        của trình biên dịch Pandora. Em sẽ đưa ra ví dụ cụ thể, như việc 
        biên dịch một chương trình in "Hello, world", sau đó thử nghiệm 
        với các đoạn mã khác để kiểm tra tính chính xác và độ ổn định của 
        compiler. Phần này cũng bao gồm việc đánh giá hiệu năng và những 
        cải tiến có thể thực hiện trong tương lai.
\end{mucluc}

Cuối cùng, báo cáo sẽ tổng kết lại những kết quả đạt được, phân tích về những
hạn chế và đề xuất một số hướng cải tiến, phát triển tiếp theo cho ngôn ngữ Pandora và 
trình biên dịch  trong tương lai, chẳng hạn như tối ưu hóa hệ thống để tăng tốc độ biên 
dịch hoặc mở rộng chức năng của ngôn ngữ Pandora, giúp ngôn ngữ lập trình Pandora có thể 
được ứng dụng thực tế tốt hơn

Chúng em hy vọng đề tài này sẽ đóng góp một phần nhỏ vào công cuộc nghiên cứu và phát 
triển các công cụ biên dịch, đồng thời tạo tiền đề cho các hướng nghiên cứu mới về 
compiler và ngôn ngữ lập trình trong tương lai.

