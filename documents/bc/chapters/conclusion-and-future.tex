\chapter*{KẾT LUẬN}
\addcontentsline{toc}{chapter}{KẾT LUẬN}

Báo cáo này đã trình bày về việc xây dựng, thiết kế một ngôn ngữ lập trình hoàn toàn mới là Pandora, cùng với trình thông dịch cho ngôn ngữ này. Pandora được thiết kế với mục tiêu giúp người dùng dễ dàng tiếp cận và sử dụng ngôn ngữ lập trình. Trình thông dịch Pandora được xây dựng với mục tiêu hỗ trợ người dùng trong việc viết và chạy chương trình, đồng thời giúp người dùng hiểu rõ hơn về ngôn ngữ Pandora thông qua việc thông báo lỗi từ vựng, cú pháp và ngữ nghĩa. Mục tiêu chính của đề tài này là tìm hiểu, nghiên cứu về ngôn ngữ lập trình và trình thông dịch, từ đó xây dựng một ngôn ngữ lập trình mới và trình thông dịch cho ngôn ngữ này. Các kết quả thực nghiệm đã chứng minh rằng Pandora đã hoạt động một cách chính xác và hiệu quả, đáp ứng được các mục tiêu, yêu cầu đã đề ra. Đề tài này cũng đã đạt được những kết quả nhất định, khi mặc dù vẫn còn đang trong quá trình phát triển và thử nghiệm, đã có gần 1800 lượt tải về và sử dụng trình thông dịch Pandora tính từ ngày 6/12/2024 đến ngày 16/12/2024. Điều này chứng tỏ rằng Pandora đã nhận được sự quan tâm và ủng hộ từ cộng đồng người dùng.

Mặc dù đã đạt được những kết quả nhất định, nhưng Pandora vẫn còn nhiều hạn chế và vấn đề cần được giải quyết. Đầu tiên và cũng là vấn đề lớn nhất đối với trình thông dịch Pandora đó chính là tốc độ chạy chương trình. Hiện tại, Pandora vẫn chưa đạt được tốc độ chạy chương trình như mong đợi, khi so sánh với các ngôn ngữ lập trình thông dịch khác như Python, Ruby, JavaScript. Điều này làm giảm trải nghiệm người dùng khi sử dụng Pandora. Tiếp theo, đó là vấn đề xử lý lỗi của trình thông dịch Pandora, tuy đã xử lý được đa dạng các loại lỗi từ vựng, cú pháp và ngữ nghĩa (khoảng trên 70 loại lỗi khác nhau), nhưng vẫn còn một số trường hợp lỗi mà trình thông dịch chưa xử lý được, hoặc xử lý nhưng vẫn chưa được tốt. Một vấn đề nữa đó chính là sự linh hoạt của ngôn ngữ Pandora, hiện tại Pandora vẫn chưa hỗ trợ đa luồng, hàm lambda, lập trình hướng đối tượng, cũng như một số tính năng khác mà người dùng mong muốn. Và cuối cùng, vấn đề về tài liệu hướng dẫn và hỗ trợ người dùng, hiện tại Pandora tuy đã có tài liệu hướng dẫn cơ bản, nhưng vẫn còn nhiều hạn chế và chưa đầy đủ, cần phải được cải thiện và bổ sung thêm.

\chapter*{HƯỚNG PHÁT TRIỂN} 
\addcontentsline{toc}{chapter}{HƯỚNG PHÁT TRIỂN}

Việc giải quyết những điểm tồn tại nói trên chính là hướng phát triển tiếp theo của Pandora. Đầu tiên là vấn đề tốc độ thực thi của trình thông dịch Pandora. Sở dĩ trình thông dịch này chạy rất chậm là do việc thông dịch diễn ra ngay trên cây cú pháp trừu tượng (AST). Việc xây dựng như này giúp giảm độ phức tạp của trình thông dịch, nhưng cũng làm tăng thời gian thực thi của chương trình do cây AST phải được duyệt qua từng node một, và một node có rất nhiều thông tin dư thừa. Một giải pháp để cải thiện vấn đề này chính là việc thêm một trình biên dịch (compiler) để chuyển đổi mã nguồn thành mã trung gian (bytecode) trước khi trình thông dịch thực thi. Mã trung gian này sẽ giúp giảm thời gian thực thi của chương trình, khi mà trình thông dịch chỉ cần duyệt qua mã trung gian này mà không cần phải xây dựng cây AST từ mã nguồn. Như vậy, Pandora sẽ có thể cải thiện tốc độ thực thi của chương trình một cách đáng kể. Nếu được, ta có thể xây trình biên dịch, dịch thẳng mã nguồn thành mã máy, giúp tăng tốc độ thực thi của chương trình lên nhiều lần. Tiếp theo, về vấn đề xử lý lỗi của trình thông dịch Pandora, cần phải tiếp tục nghiên cứu và phát triển thêm để xử lý được nhiều loại lỗi hơn, cũng như cải thiện chất lượng thông báo lỗi để người dùng dễ dàng hiểu hơn. Về vấn đề sự linh hoạt của ngôn ngữ Pandora, cần phải nghiên cứu và phát triển thêm các tính năng mới như đa luồng, hàm lambda, lập trình hướng đối tượng, cũng như một số tính năng khác mà người dùng mong muốn. Cuối cùng, về vấn đề tài liệu hướng dẫn và hỗ trợ người dùng, cần phải cải thiện và bổ sung thêm tài liệu hướng dẫn, cũng như cung cấp các kênh hỗ trợ khác như diễn đàn, nhóm hỗ trợ, email hỗ trợ, để người dùng có thể dễ dàng tìm kiếm thông tin và giải đáp thắc mắc.
