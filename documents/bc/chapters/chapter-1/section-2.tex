\section{Khái niệm và vai trò của compiler}
\subsection{Khái niệm về compiler}
Chương trình dịch là 1 chương trình dùng để
chuyển 1 chương trình từ ngôn ngữ này (ngôn ngữ nguồn) thành 1 chương
trình tương đương ở ngôn ngữ khác (ngôn ngữ đích).
\subsection{Vai trò của compiler}
Trong quá trình dịch, chương trình dịch có 
thể phát hiện lỗi cú pháp, ngữ nghĩa, đồng thời có thể tối ưu hóa chương
trình (giảm thiểu số câu lệnh, tôi ưu bộ nhớ,...). Ngoài ra, trình biên
giúp cho các chương trình nguồn độc lập với phần cứng, chạy được trên
nhiều hệ thống khác nhau.