\subsection{Phân tích ngữ nghĩa và thực thi}

Giai đoạn phân tích ngữ nghĩa và thực thi được thực hiện trên từng câu lệnh của chương trình nguồn. Giai đoạn này bao gồm hai giai đoạn nhỏ được thực hiện lần lượt trên từng câu lệnh là: Phân tích ngữ nghĩa và thực thi. Cụ thể với mỗi câu lệnh, từng giai đoạn nhỏ sẽ có nhiệm vụ như sau:

\textbf{Giai đoạn 1.} Phân tích ngữ nghĩa: 

Giai đoạn này kiểm tra xem câu lệnh trong chương trình nguồn có đúng theo các quy định về ngữ nghĩa của ngôn ngữ nguồn hay không, đảm bảo sẽ phát hiện và mô tả các lỗi liên quan. Việc kiểm tra này có thể bao gồm (có thể sẽ được mở rộng khi ngôn ngữ phát triển thêm về sau): 
\begin{itemize}
    \item Kiểm tra kiểu. Trình thông dịch sẽ kiểm tra để đảm bảo một biểu thức phải được thực hiện trên các toán tử có kiểu hợp lệ, các câu lệnh phải được thực hiện với các biểu thức có kiểu tương thích với câu lệnh hoặc khi cố gắng ép kiểu các kiểu dữ liệu phải hợp lệ và có thể ép kiểu.
    \item Kiểm tra dòng điều khiển. Có một số câu lệnh thay đổi dòng điều khiển, khiến chương trình thay đổi thứ tự thực hiện các câu lệnh như  \kw{break}, \kw{continue}, \dots (Trong Pandora tương ứng là các câu lệnh \kw{br}, \kw{skip}). Trình thông dịch sẻ kiểm tra để đảm bảo các câu lệnh này nằm bên trong một vòng lặp không.
    \item Kiểm tra tính nhất quán. Trình thông dịch sẽ kiểm tra những trường hợp như các biến được tạo với kiểu không được thay đổi giá trị sẽ chỉ được gán giá trị một lần duy nhất, các hàm sẽ chỉ được định nghĩa một lần, \dots
    \item Kiểm tra phạm vi. Trình thông dịch sẽ kiểm tra một biến, hàm, \dots có được định nghĩa đúng phạm vi hay không trước khi thực hiện các câu lệnh trên chúng.
\end{itemize}

\textbf{Giai đoạn 2.} Thực thi: 

Sau khi kiểm tra ngữ nghĩa thành công, trình thông dịch thực thi câu lệnh và lưu kết quả vào bộ nhớ hoặc trả về đầu ra.
