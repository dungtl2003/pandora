\subsection{Xử lý lỗi}
    Nếu một trình thông dịch chỉ phải dịch các chương trình nguồn viết đúng thì thiết kế và hoạt động của nó sẽ rất đơn giản. Nhưng những người lập trình lại thường xuyên tạo ra những chương trình viết sai. Một trình thông dịch tốt phải phát hiện, định vị, phân loại được tất cả các lỗi để giúp đỡ người viết. Do vậy, việc thiết kế và thực hiện một chương trình dịch có phần xử lý lỗi tốt trở thành một thách thức lớn.

    Các lỗi có thể xuất hiện ở bất kỳ giai đoạn nào, tùy theo mức độ nghiêm trọng của lỗi thì trình thông dịch sẽ phân biệt lỗi đó có thể khôi phục hay không, có mỗi số lỗi khi gặp sẽ dừng ngay chương trình. Mỗi giai đoạn đều có cách xử lý lỗi khác nhau. Chẳng hạn, như ở giai đoạn thông dịch, nếu bộ xử lý gặp một lỗi nào đó bất kì thì nó đều yêu cầu dừng ngay chương trình và thông báo lỗi. Còn ở giai đoạn phân tích cú pháp, thay vì dừng ngay chương trình, bộ xử lý sẽ đi vào trạng thái \textbf{hoảng loạn} (panic) và cố gắng phục hồi chương trình về trạng thái bình thường (tất cả sẽ được phân tích chi tiết ở phần \textbf{\ref{ch3:err-handler}}).

    Công việc xử lý lỗi đóng một vai trò rất to lớn trong quá trình thông dịch. Một trình thông dịch xử lý lỗi tốt sẽ giúp người dùng có thể phát hiện lỗi một cách nhanh chóng và xử lý lỗi ngay lập tức, từ đó giúp tăng năng suất người dùng. 
