\section{Giới thiệu về ngôn ngữ Pandora}
Nguồn gốc tên Pandora: 
    Tên gọi Pandora bắt nguồn từ thần thoại 
    Hy Lạp cổ đại. Trong câu chuyện, Pandora là người phụ nữ đầu tiên được 
    các vị thần tạo ra, người đã mở chiếc hộp chứa đựng mọi điều xấu xa tràn 
    vào thế giới, nhưng đồng thời hy vọng vẫn còn ở lại trong đó.

Ý nghĩa và lí do chọn Pandora để đặt tên cho ngôn ngữ: 
    Chiếc hộp Pandora là biểu tượng của sự tò mò, những điều không thể 
    lường trước, và cũng là khởi đầu cho những thay đổi to lớn. Nhóm chúng 
    em chọn tên Pandora cho ngôn ngữ lập trình mới này để truyền tải ý 
    tưởng về những thử thách, khó khăn và rủi ro tiềm ẩn khi khám phá một 
    ngôn ngữ hoàn toàn mới. Tuy nhiên, ẩn sau đó cũng là hy vọng cho một 
    khởi đầu mới mẻ, một hành trình đi tìm tri thức và giải pháp sáng tạo.

Ngôn ngữ lập trình Pandora cho phép viết các biểu thức số học, bao gồm các phép toán số học thông thường như cộng (+), trừ (-), nhân (*), chia (/), chia lấy dư (\%), và các phép toán logic như AND, OR, NOT, XOR, \dots Ngoài ra, Pandora cũng hỗ trợ các phép dịch bit như dịch trái ({<}{<}) và dịch phải ({>}{>}), giúp lập trình viên thực hiện các thao tác trên dữ liệu nhị phân một cách hiệu quả. Ngoài ra ngôn ngữ Pandora cũng hỗ trợ đầy đủ các toán tử so sánh như bằng (==), không bằng (!=), lớn hơn (>), nhỏ hơn (<), lớn hơn hoặc bằng (>=), nhỏ hơn hoặc bằng (<=), AND bằng (\&=), \dots Điều này giúp lập trình viên thực hiện các phép so sánh và kiểm tra điều kiện một cách dễ dàng và trực quan.

Pandora là một ngôn ngữ lập trình hiện đại với cấu trúc linh hoạt. Ngôn ngữ này có các câu lệnh khai báo như biến, hằng số, hàm, lớp và giao diện, cho phép tổ chức và quản lý mã nguồn một cách rõ ràng và dễ hiểu. Bên cạnh đó, Pandora có các câu lệnh điều khiển như if, while, và for, giúp lập trình viên dễ dàng xử lý các luồng điều kiện và vòng lặp phức tạp.
    
Các biểu thức trong Pandora rất đa dạng, từ các biểu thức tính toán đơn giản cho đến các phép toán logic và điều kiện. Pandora đảm bảo cung cấp đủ các loại phép toán và cấu trúc câu lệnh cần thiết để lập trình viên có thể xây dựng những chương trình phức tạp và hiệu quả.

Phạm vi hoạt động (scope) của biến và hằng trong Pandora được quản lý chặt chẽ để tránh xung đột và tăng tính bảo mật cho mã nguồn. Cấu trúc chương trình trong Pandora được thiết kế với mục tiêu giữ cho mã nguồn rõ ràng và đơn giản nhất có thể, giúp việc biên dịch và tối ưu hóa chương trình trở nên dễ dàng hơn.