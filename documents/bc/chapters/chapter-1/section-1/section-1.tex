\section{Khái niệm và vai trò của trình thông dịch}

Trình thông dịch (Interpreter) là một chương trình máy tính có nhiệm vụ thực thi trực tiếp mã nguồn của một ngôn ngữ lập trình mà không cần biên dịch trước thành mã máy (machine code).

Trình thông dịch hoạt động theo nguyên tắc "dịch và thực thi từng phần" (read-evaluate-execute). Điều này làm cho nó khác biệt so với trình biên dịch (compiler), vốn dịch toàn bộ mã nguồn thành mã máy trước khi chạy.

Trình thông dịch thực hiện việc đọc mã nguồn, phân tích, và thực thi từng câu lệnh ngay lập tức. Trong quá trình này, nó có thể phát hiện lỗi trong từng giai đoạn. Ngoài ra, trình thông dịch giúp cho các chương trình nguồn độc lập với phần cứng, chạy được trên nhiều hệ thống khác nhau.

% \subsection{Từ tố (\textit{token})}
    Những từ tố đơn giản được định nghĩa bằng cách đưa trực tiếp mẫu của
nó (như các định nghĩa về từ khoá, dấu toán học, ...). Những từ tố phức tạp
hơn sẽ được định nghĩa bằng biểu thức chính quy như tên, xâu, số và chú thích.
Phần phân tích từ vựng (lexical analyzer) phải có khả năng phân tích chương trình nguồn
và đưa ra được các từ tố như sau:

\chapter*{BẢNG KÝ HIỆU}
\addcontentsline{toc}{chapter}{BẢNG KÝ HIỆU}
\begin{longtable}{| c | c |}
\hline
\textbf{\textit{Từ viết tắt}} & \textbf{\textit{Từ gốc}} \\
\hline
\end{longtable}


    Từ vựng của Pandora được xác định không chỉ từ chữ cái (alphabet) và chữ số (digit), mà là hầu hết các ký tự trong bảng mã Unicode \cite{allen2012unicode}, cụ thể như sau:

    \regexdigit

    \regexalphabet

\noindent và các ký tự khác trong bảng mã Unicode.

    Khoảng trắng (\textbf{whitespace}) là bất kỳ chuỗi không trống nào chỉ chứa các ký tự có thuộc tính Unicode Pattern\_White\_Space \cite{web:unicode:report}, cụ thể là:
    \begin{itemize}
        \item{U+0009 (horizontal tab, '\textbackslash t')}
        \item{U+000A (line feed, '\textbackslash n')}
        \item{U+000B (vertical tab)}
        \item{U+000C (form feed)}
        \item{U+000D (carriage return, '\textbackslash r')}
        \item{U+0020 (space, ' ')}
        \item{U+0085 (next line)}
        \item{U+200E (left-to-right mark)}
        \item{U+200F (right-to-left mark)}
        \item{U+2028 (line separator)}
        \item{U+2029 (paragraph separator)}
    \end{itemize}
\noindent Pandora là một ngôn ngữ "dạng tự do", có nghĩa là tất cả các dạng khoảng trắng chỉ dùng để phân tách các từ tố trong ngữ pháp và không có ý nghĩa ngữ nghĩa.

    Tên \textbf{\textit{identifier}} trong Pandora được phân làm 2 loại: \textbf{non keyword identifier} hoặc \textbf{raw identifier}. \textbf{non keyword identifier} được tạo thành từ tập các kí tự nêu trên và không được là từ khóa (keyword). Trong khi đó, \textbf{raw identifier} có thể là tên hoặc từ khóa (có \textbf{r\#} ở phía trước để phân biệt với tên thường hay từ khóa), cụ thể như sau:

    \regexidentifier

\noindent Trong đó \textbf{XID\_Start} và \textbf{XID\_Continue} là các thuộc tính của ký tự trong Unicode liên quan đến tên (định danh). Chúng thường được sử dụng để xác định liệu một ký tự có thể là phần đầu hoặc phần thân của 1 định danh trong các ngôn ngữ lập trình hay không. Ngoài ra, 1 tên còn có thể chứa các biểu tượng cảm xúc (\textbf{EMOJI\_SYMBOL}).

    1 ký tự chữ (\textbf{character}) là 1 ký tự Unicode đơn được đặt trong 2 ký tự ' (dấu nháy đơn - U+0027), ngoại trừ chính U+0027, ký tự này phải được thoát bằng ký tự U+005C trước đó (\textbackslash), cụ thể như sau:

    \regexcharliteral

    Chuỗi ký tự (\textbf{string literal}) là một chuỗi gồm bất kỳ ký tự Unicode nào được đặt trong hai ký tự U+0022 (dấu ngoặc kép), ngoại trừ chính U+0022, ký tự này phải được thoát bằng ký tự U+005C trước đó (\textbackslash). Chuỗi ký tự cho phép có các line-breaks (cho phép ngắt dòng, được biểu thị bởi ký tự U+000A). Khi ký tự U+005C không thoát (\textbackslash) xuất hiện ngay trước ngắt dòng, ngắt dòng sẽ không xuất hiện trong xâu được biểu diễn trong từ tố, cụ thể như sau:

    \regexstringliteral

    Ta có thể thấy, các ký tự chữ hoặc các xâu có thể chứa 1 hoặc 1 vài loại \textbf{escape} (thoát ký tự). Một escape bắt đầu bằng U+005C (\textbackslash) và tiếp tục bằng một trong các dạng sau:

    \begin{itemize}
    \item{\textbf{Whitespace escape} (thoát khoảng trắng) là 1 trong các ký tự U+006E (n), U+0072 (r) hoặc U+0074 (t), biểu thị các giá trị Unicode U+000A (LF), U+000D (CR) hoặc U+ 0009 (HT) tương ứng}
    \item{\textbf{Null escape} (thoát null) là ký tự U+0030 (0) và biểu thị giá trị Unicode U+0000 (NUL)}
    \item{\textbf{Backslash escape} (thoát gạch chéo ngược) là ký tự U+005C (\textbackslash) phải được thoát để biểu thị chính nó}
    \end{itemize}

    Chuỗi ký tự thô (\textbf{raw string literal}) không xử lý bất kỳ escape nào. Chúng bắt đầu bằng ký tự U+0072 (r), theo sau là ít hơn 256 ký tự U+0023 (\#) và ký tự U+0022 (dấu nháy kép). Raw string có thể chứa chuỗi bất kì các ký tự nào trong Unicode. Nó chỉ được kết thúc bởi một ký tự U+0022 (dấu nháy kép) khác, theo sau là các ký tự U+0023 (\#) có số lượng giống với các ký tự U+0023 (\#) đứng trước ký tự U+0022 (dấu nháy kép) mở đầu. Tất cả các ký tự Unicode đều có trong phần thân raw string thể hiện chính chúng, các ký tự U+0022 (dấu nháy kép) (trừ khi được theo sau bởi ít nhất nhiều ký tự U+0023 (\#) như đã được sử dụng để bắt đầu raw string) hoặc U+005C (\textbackslash) không có ý nghĩa gì đặc biệt. Cụ thể như sau:

    \regexrawstringliteral

    Một hằng số nguyên (\textbf{integer literal}) có một trong bốn dạng sau:

    \begin{itemize}
        \item{Một hằng thập phân bắt đầu bằng một chữ số thập phân và tiếp tục bằng bất kỳ hỗn hợp nào của các chữ số thập phân và dấu gạch dưới}
        \item{Một hằng thập lục phân bắt đầu bằng chuỗi ký tự U+0030 U+0078 (0x) và tiếp tục dưới dạng bất kỳ hỗn hợp nào (có ít nhất một chữ số) gồm các chữ số thập lục phân và dấu gạch dưới}
        \item{Một chữ bát phân bắt đầu bằng chuỗi ký tự U+0030 U+006F (0o) và tiếp tục dưới dạng bất kỳ hỗn hợp nào (có ít nhất một chữ số) gồm các chữ số bát phân và dấu gạch dưới}
        \item{Một chữ số nhị phân bắt đầu bằng chuỗi ký tự U+0030 U+0062 (0b) và tiếp tục dưới dạng bất kỳ hỗn hợp nào (có ít nhất một chữ số) gồm các chữ số nhị phân và dấu gạch dưới}
    \end{itemize}

    \regexintegerliteral

    Một hằng số thực (\textbf{float literal}) có một trong hai dạng:
    \begin{itemize}
        \item{Một số thập phân theo sau là ký tự dấu chấm U+002E (.). Theo sau có thể là một số thập phân khác (sau số thập phân đó có thể có số mũ}
        \item{Một số thập phân theo sau là số mũ}
    \end{itemize}

    \regexfloatliteral

    Chú thích không phải tài liệu (\textbf{non-doc comment}) có thể là dòng (//) hoặc là khối (/* ... */). Pandora có hỗ trợ các chú thích khối lồng nhau. Các chú thích kiểu này được hiểu là một dạng khoảng trắng.

    Chú thích tài liệu (\textbf{doc comment}) được chia làm 2 loại chính: chú thích tài liệu ngoài (\textbf{outer doc comment}) và chú thích tài liệu trong (\textbf{inner doc comment}). Chú thích tài liệu dòng bên ngoài sẽ bắt đầu bằng //@, còn khối bên ngoài sẽ có dạng /*@ ... */. Trong khi đó, chú thích tài liệu dòng bên trong là //!, khối bên trong có dạng /*! ... */.

    Các chú thích được biểu diễn như sau:

    \regexlinecomment

    \regexblockcomment

    \regexdoc


% \subsection{Phân tích cú pháp}
\subsubsection{Mục đích}
Phân tích cú pháp là một giai đoạn quan trọng trong quá trình xây dựng compiler, với mục tiêu chuyển đổi chuỗi các token từ bộ phân tích từ vựng (lexer) thành một cấu trúc cây cú pháp trừu tượng (AST). Cây AST này đại diện cho cấu trúc logic của chương trình, giúp xác định cách các phần tử trong mã nguồn liên kết với nhau và hỗ trợ quá trình phân tích ngữ nghĩa cũng như sinh mã.

\vspace{1cm}
\hspace{-1.5cm}
\begin{figure}[h]
    % \centering
    \begin{tikzpicture}[
        % roundnode/.style={circle, draw=green!60, fill=green!5, very thick, minimum size=7mm},
        squarednode/.style={rectangle, draw=red!60, fill=red!5, very thick, minimum size=5mm},
        ]
    
        %Nodes
        \node[squarednode,text width=2.5cm,align=center](lexer){Phân tích từ vựng};
        \node[text width=2cm,align=center](source)[above=of lexer, xshift = -2.5cm, yshift = -1.5cm]{Chương trình nguồn};
        \node[](nothingsource)[left=of lexer]{};
    
        \node[squarednode,text width=3cm,align=center](parser)[right=of lexer,xshift = 0.5cm] {Phân tích cú pháp};
        \node[squarednode,text width=3cm,align=center](tableofsymbols)[below=of parser, yshift = -0.5cm] {Bảng ký hiệu};
        \node[squarednode,text width=2.5cm,align=center](semanticanalysis)[right=of parser,xshift=0.5cm]{Phân tích ngữ nghĩa}; 
    
        % \node[text width=2cm,align=center](textabove)[below=of lexer,xshift = 2.5cm, yshift = 1.5cm]{Yêu cầu lấy từ tố tiếp theo}; 
        \node[text width=2cm,align=center](textbelow)[above=of lexer,xshift = 2.2cm, yshift = -1.5cm]{Từ tố}; 
        \node[text width=1.5cm,align=center](textaboveparser)[above=of parser,xshift = 2.5cm, yshift = -1.5cm]{Cây phân tích}; 

        \node[](nothing)[right=of semanticanalysis]{};
    
        \coordinate (A) at (lexer.south);
        \coordinate (B) at ([yshift = 2mm]tableofsymbols.west);
    
    
        %Lines
        \draw[->] (nothingsource) -- (lexer);
        \draw[->] (lexer.east) -- (parser.west);
        \draw[<->] (lexer.south) |- ([yshift = 2mm]tableofsymbols.west);
        % \draw[->] ([yshift = -2mm]parser.west) -- ([yshift = -2mm]lexer.east);
        \draw[<->] (tableofsymbols.east)  -| (semanticanalysis.south);
        \draw[->] (parser.east) -- (semanticanalysis.west); %use dashed for --->
        \draw[->] (semanticanalysis.east) -- (nothing.west);
    
    \end{tikzpicture}
    \caption{Vị trí của bộ phân tích cú pháp.}
    \label{fig:pos-of-parser}
\end{figure}
% \begin{tikzpicture}[
%     % roundnode/.style={circle, draw=green!60, fill=green!5, very thick, minimum size=7mm},
%     squarednode/.style={rectangle, draw=red!60, fill=red!5, very thick, minimum size=5mm},
%     ]

%     %Nodes
%     \node[squarednode,text width=2.5cm,align=center](lexer){Phân tích từ vựng};
%     \node[text width=2cm,align=center](source)[above=of lexer, xshift = -2.5cm, yshift = -1.5cm]{Chương trình nguồn};
%     \node[](nothingsource)[left=of lexer]{};

%     \node[squarednode,text width=3cm,align=center](parser)[right=of lexer,xshift = 1cm] {Phân tích cú pháp};
%     \node[squarednode,text width=3cm,align=center](tableofsymbols)[below=of lexer, xshift = 3cm, yshift = -0.5cm] {Bảng ký hiệu};
%     \node[squarednode,text width=2.5cm,align=center](semanticanalysis)[right=of parser]{Phân tích ngữ nghĩa}; 

%     \node[text width=2cm,align=center](textabove)[above=of lexer,xshift = 2.5cm, yshift = -1.2cm]{Yêu cầu lấy từ tố tiếp theo}; 
%     \node[text width=2cm,align=center](textbelow)[below=of lexer,xshift = 2.5cm, yshift = 1.2cm]{Từ tố}; 
%     \node[](nothing)[right=of semanticanalysis]{};

%     \coordinate (A) at (lexer.south);
%     \coordinate (B) at ([yshift = 2mm]tableofsymbols.west);


%     %Lines
%     \draw[->] (nothingsource) -- (lexer);
%     \draw[->] ([yshift = 2mm]lexer.east) -- ([yshift = 2mm]parser.west);
%     \draw[<->] (lexer.south) |- ([yshift = 2mm]tableofsymbols.west);
%     \draw[->] ([yshift = -2mm]parser.west) -- ([yshift = -2mm]lexer.east);
%     \draw[<->] (tableofsymbols.east)  -| (semanticanalysis.south);
%     \draw[->] (parser.east) -- (semanticanalysis.west); %use dashed for --->
%     \draw[->] (semanticanalysis.east) -- (nothing.west);

% \end{tikzpicture}
\vspace{1cm}

Mục đích chính của phân tích cú pháp là đảm bảo mã nguồn tuân thủ đúng các quy tắc cú pháp của ngôn ngữ lập trình. Qua đó, compiler có thể phát hiện và báo cáo các lỗi cú pháp, giúp nhà phát triển sửa chữa trước khi chuyển sang các giai đoạn xử lý tiếp theo. Phân tích cú pháp đóng vai trò như một bước đệm giữa việc nhận diện token và việc kiểm tra ý nghĩa logic cũng như sinh mã cuối cùng, giúp đảm bảo quá trình biên dịch diễn ra mượt mà và chính xác.
\subsubsection{Một số phương pháp phân tích cú pháp}
Ở đây chúng em đề cập đến hai thuật toán phân tích phổ biến nhất là \textit{Thuật toán phân tích Top-down} và \textit{Thuật toán phân tích Bottom-up}

\textbf{Thuật toán phân tích Top-down:}
Tên \textit{phân tích Top-down} xuất phát từ ý tưởng cố gắng tạo ra một cây phân tích cho đầu vào bắt đầu từ đỉnh và đi xuống cho đến lá.

\textbf{Thuật toán phân tích Bottom-up:}
Phương pháp \textit{phân tích Bottom-up} về tư tưởng là ngược lại với phương pháp \textit{phân tích Bottom-up}. Phương pháp này lại bắt đầu từ lá (tức là từ chính các ký hiệu đầu vào) và cố gắng xây dựng thành cây bằng cách hướng lên gốc.

\subsubsection{Liên hệ với trình biên dịch ngôn ngữ Pandora}
Trong thiết kế và xây dựng bộ phân tích cú pháp cho trình biên dịch ngôn ngữ Pandora, chúng em sử dụng thuật toán phân tích Top-down và cụ thể là sử dụng phương pháp phân tích đệ quy đi xuống (recursive descent parsing).

\textit{Phân tích đệ quy đi xuống} là một kỹ thuật phân tích cú pháp từ trên xuống, xây dựng cây cú pháp từ đỉnh và đọc đầu vào từ trái sang phải. Phương pháp này sử dụng các thủ tục tương ứng cho từng từ tố đơn (\textit{\emph{Các kí tự đơn như} Colon, Comma, Plus, Minus \dots \emph{hoặc các từ khóa như} if, for, \dots }) và các ký hiệu không kết thúc (\textit{IF\_STATEMENT, LOOP\_STATEMENT, BLOCK\_STATEMENT, \dots}), với mỗi \\quy tắc ngữ pháp được triển khai dưới dạng một hàm hoặc thủ tục riêng biệt. Quá trình phân tích diễn ra đệ quy, nghĩa là các hàm gọi lại chính chúng hoặc gọi các hàm khác để phân tích các phần tử con.
% \cite{} TODO!

\textbf{Lý do chọn phương pháp này:} Với ngôn ngữ không quá phức tạp như ngôn ngữ Pandora do chúng em thiết kế, phương pháp đệ quy xuống là một lựa chọn hợp lý để xây dựng trình biên dịch. Đây là kỹ thuật phân tích cú pháp từ trên xuống, giúp mô phỏng cấu trúc ngữ pháp của ngôn ngữ một cách trực quan và dễ hiểu. Phương pháp này không chỉ dễ dàng triển khai mà còn rất thuận tiện cho việc bảo trì và mở rộng hệ thống sau này. Bởi vì nó cho phép phân tích cú pháp theo từng phần tử nhỏ, từ đó dễ dàng kiểm tra và sửa lỗi. Hơn nữa, với ngữ pháp đơn giản của Pandora, việc áp dụng đệ quy xuống giúp giảm bớt độ phức tạp trong việc viết mã và tối ưu hóa hiệu suất của trình biên dịch.
