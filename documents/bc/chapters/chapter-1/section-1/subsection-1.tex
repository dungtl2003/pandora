\subsection{Khái niệm về trình thông dịch}
Trình thông dịch (Interpreter) là một chương trình máy tính có nhiệm vụ thực thi trực tiếp mã nguồn của một ngôn ngữ lập trình mà không cần biên dịch trước thành mã máy (machine code). Nó thực hiện việc đọc mã nguồn, phân tích, kiểm tra lỗi, và thực thi từng câu lệnh ngay lập tức.

Trình thông dịch hoạt động theo nguyên tắc "dịch và thực thi từng phần" (read-evaluate-execute). Điều này làm cho nó khác biệt so với trình biên dịch (compiler), vốn dịch toàn bộ mã nguồn thành mã máy trước khi chạy.