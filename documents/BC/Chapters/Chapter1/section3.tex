\section{Giới thiệu về ngôn ngữ Rust}
Rust là một ngôn ngữ lập trình hệ thống hiện đại, nổi bật với khả năng đảm bảo hiệu năng cao, tính an toàn bộ nhớ, và sự quản lý tài nguyên tối ưu. Với sự kết hợp giữa tốc độ tương đương C/C++ và các tính năng hiện đại như quản lý bộ nhớ an toàn mà không cần garbage collector, Rust mang lại lợi thế rõ rệt trong việc phát triển những phần mềm đòi hỏi độ tin cậy và hiệu suất cao. Ngoài ra, Rust có hệ thống kiểu mạnh mẽ, giúp hạn chế lỗi tiềm ẩn ngay từ khâu biên dịch. Điều này đặc biệt quan trọng trong việc xây dựng trình biên dịch – một loại phần mềm cần xử lý chính xác và hiệu quả các thao tác liên quan đến cú pháp, ngữ nghĩa, và quản lý tài nguyên. 

Lý do chọn Rust làm ngôn ngữ cho trình biên dịch: 
\vspace{-10pt}
\begin{itemize}
\setlength\itemsep{-5pt}
\item Hiệu năng cao: Trình biên dịch thường cần xử lý lượng lớn dữ liệu nhanh chóng, và Rust có khả năng tối ưu hóa tương tự C/C++. 
\item An toàn bộ nhớ: Việc quản lý tài nguyên trong Rust giúp giảm thiểu các lỗi phổ biến như tràn bộ nhớ hoặc lỗi truy cập null pointer, điều này rất cần thiết cho sự ổn định của trình biên dịch. 
\item Cộng đồng và thư viện phong phú: Rust có hệ sinh thái mạnh với nhiều thư viện hữu ích hỗ trợ phân tích cú pháp, giúp đẩy nhanh quá trình phát triển trình biên dịch.      
\end{itemize}
\vspace{-10pt}

Với sự kết hợp giữa hiệu năng, độ tin cậy và khả năng mở rộng, Rust là sự lựa chọn lý tưởng cho việc phát triển trình biên dịch cho ngôn ngữ mới của chúng em. 